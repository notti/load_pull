\documentclass[12pt,a4paper,parskip=full,abstract=true,BCOR=10mm,twoside,open=right]{scrreprt}
\KOMAoption{bibliography}{totoc}
\KOMAoption{listof}{totoc}

\usepackage[ngerman,english]{babel}
\usepackage[utf8]{inputenc}

%drawings

\usepackage[pdftex]{graphicx}
\graphicspath{{graphics/}}
\usepackage{tikz}
\usepackage{circuitikz}
\usetikzlibrary{positioning,dsp,chains,fit,scopes,calc,backgrounds,arrows,decorations.pathmorphing,bending,arrows.meta,shapes.misc,matrix,patterns}

\makeatletter

\ctikzset{rf/width/.initial=0.75}
\ctikzset{rf/height/.initial=0.75}
\ctikzset{rf/filter/sinewidth/.initial=0.8}
\ctikzset{rf/filter/sineheight/.initial=0.2}
\ctikzset{rf/switch/length/.initial=0.6}


\long\def\pgfrfdeclarenode#1#2#3{
    \pgfdeclareshape{#1}
    {
        \anchor{center}{
            \pgfpointorigin
        }
        \savedanchor\northwest{%
            \pgf@y=\pgfkeysvalueof{/tikz/circuitikz/bipoles/length}
            \pgf@y=\pgfkeysvalueof{/tikz/circuitikz/rf/height}\pgf@y
            \pgf@y=.5\pgf@y
            \pgf@x=\pgfkeysvalueof{/tikz/circuitikz/bipoles/length}
            \pgf@x=.5\pgf@x
            \pgf@x=-\pgfkeysvalueof{/tikz/circuitikz/rf/width}\pgf@x
        }
        \anchor{north}{
            \northwest
            \pgf@x=0pt
        }
        \anchor{south}{
            \northwest
            \pgf@x=0pt
            \pgf@y=-\pgf@y
        }
        \anchor{west}{
            \northwest
            \pgf@y=0pt
        }
        \anchor{east}{
            \northwest
            \pgf@y=0pt
            \pgf@x=-\pgf@x
        }
        \anchor{south west}{
            \northwest
            \pgf@y=-\pgf@y
        }
        \anchor{north east}{
            \northwest
            \pgf@x=-\pgf@x
        }
        \anchor{north west}{
            \northwest
        }
        \anchor{south east}{
            \northwest
            \pgf@x=-\pgf@x
            \pgf@y=-\pgf@y
        }	  
        \anchor{base}{
            \northwest
            \pgf@x=0pt	  	
        }
        \anchorborder{
            \@tempdima=\pgf@x
            \@tempdimb=\pgf@y
            \northwest
            \pgf@xa=-\pgf@x
            \pgf@ya=\pgf@y
            \pgfpointborderrectangle{\pgfpoint{\@tempdima}{\@tempdimb}}{\pgfpoint{\pgf@xa}{\pgf@ya}}
        }
        #3
        \backgroundpath{			
            \pgfsetcolor{\pgfkeysvalueof{/tikz/circuitikz/color}}	

            \northwest
            \pgf@circ@res@up = \pgf@y 
            \pgf@circ@res@down = -\pgf@y
            \pgf@circ@res@right = -\pgf@x
            \pgf@circ@res@left = \pgf@x

            \pgfscope
                \pgfsetcornersarced{\pgfpoint{4pt}{4pt}}
                \pgfpathrectanglecorners{\pgfpoint{\pgf@circ@res@left}{\pgf@circ@res@down}}{\pgfpoint{\pgf@circ@res@right}{\pgf@circ@res@up}}
                \pgfstroke
            \endpgfscope

            #2

        }
    }
}

\long\def\pgfrfdeclaresimplenode#1#2#3#4{
    \pgfdeclareshape{#1}
    {
        \anchor{center}{
            \pgfpointorigin
        }
        \savedanchor\northwest{%
            #2
        }
        \anchor{north}{
            \northwest
            \pgf@x=0pt
        }
        \anchor{south}{
            \northwest
            \pgf@x=0pt
            \pgf@y=-\pgf@y
        }
        \anchor{west}{
            \northwest
            \pgf@y=0pt
        }
        \anchor{east}{
            \northwest
            \pgf@y=0pt
            \pgf@x=-\pgf@x
        }
        \anchor{south west}{
            \northwest
            \pgf@y=-\pgf@y
        }
        \anchor{north east}{
            \northwest
            \pgf@x=-\pgf@x
        }
        \anchor{north west}{
            \northwest
        }
        \anchor{south east}{
            \northwest
            \pgf@x=-\pgf@x
            \pgf@y=-\pgf@y
        }	  
        \anchor{base}{
            \northwest
            \pgf@x=0pt	  	
        }
        #3
        \backgroundpath{			
            #4
        }
    }
}

\long\def\pgfrfdeclaredoublenode#1#2#3{
    \pgfdeclareshape{#1}
    {
        \anchor{center}{
            \pgfpointorigin
        }
        \savedanchor\northwest{%
            \pgf@y=\pgfkeysvalueof{/tikz/circuitikz/bipoles/length}
            \pgf@y=\pgfkeysvalueof{/tikz/circuitikz/rf/height}\pgf@y
            \pgf@y=.5\pgf@y
            \pgf@x=\pgfkeysvalueof{/tikz/circuitikz/bipoles/length}
            \pgf@x=.75\pgf@x
            \pgf@x=-\pgfkeysvalueof{/tikz/circuitikz/rf/width}\pgf@x
        }
        \anchor{north}{
            \northwest
            \pgf@x=0pt
        }
        \anchor{south}{
            \northwest
            \pgf@x=0pt
            \pgf@y=-\pgf@y
        }
        \anchor{west}{
            \northwest
            \pgf@y=0pt
        }
        \anchor{east}{
            \northwest
            \pgf@y=0pt
            \pgf@x=-\pgf@x
        }
        \anchor{south west}{
            \northwest
            \pgf@y=-\pgf@y
        }
        \anchor{north east}{
            \northwest
            \pgf@x=-\pgf@x
        }
        \anchor{north west}{
            \northwest
        }
        \anchor{south east}{
            \northwest
            \pgf@x=-\pgf@x
            \pgf@y=-\pgf@y
        }	  
        \anchor{base}{
            \northwest
            \pgf@x=0pt	  	
        }
        \anchorborder{
            \@tempdima=\pgf@x
            \@tempdimb=\pgf@y
            \northwest
            \pgf@xa=-\pgf@x
            \pgf@ya=\pgf@y
            \pgfpointborderrectangle{\pgfpoint{\@tempdima}{\@tempdimb}}{\pgfpoint{\pgf@xa}{\pgf@ya}}
        }
        #3
        \backgroundpath{			
            \pgfsetcolor{\pgfkeysvalueof{/tikz/circuitikz/color}}	

            \northwest
            \pgf@circ@res@up = \pgf@y 
            \pgf@circ@res@down = -\pgf@y
            \pgf@circ@res@right = -\pgf@x
            \pgf@circ@res@left = \pgf@x

            \pgfscope
                \pgfsetcornersarced{\pgfpoint{4pt}{4pt}}
                \pgfpathrectanglecorners{\pgfpoint{\pgf@circ@res@left}{\pgf@circ@res@down}}{\pgfpoint{\pgf@circ@res@right}{\pgf@circ@res@up}}
                \pgfstroke
            \endpgfscope

            #2

        }
    }
}

\long\def\pgfrfdeclaremonopole#1#2{
    \pgfrfdeclarenode{#1}{#2}{
        \anchor{A}{
            \northwest
            \pgf@y=0pt
        }
    }
}

\long\def\pgfrfdeclarebipole#1#2{
    \pgfrfdeclarenode{#1}{#2}{
        \anchor{A}{
            \northwest
            \pgf@y=0pt
        }
        \anchor{B}{
            \northwest
            \pgf@y=0pt
            \pgf@x=-\pgf@x
        }
    }
}

\long\def\pgfrfdeclarebipoleslash#1#2{
    \pgfrfdeclarebipole{#1}{
        \pgf@circ@res@step = \pgf@circ@res@left
        \pgfmathaddtolength{\pgf@circ@res@step}{4pt - 4pt*cos(-45)}
        \pgfpathmoveto{\pgfpoint{\pgf@circ@res@step}{\pgf@circ@res@step}}
        \pgfpathlineto{\pgfpoint{-\pgf@circ@res@step}{-\pgf@circ@res@step}}
        \pgfusepath{draw}
        #2
    }
}

\long\def\pgfrfdeclaretripole#1#2{
    \pgfrfdeclarenode{#1}{#2}{
        \anchor{A1}{
            \northwest
            \pgf@y=0pt
        }
        \anchor{B1}{
            \northwest
            \pgf@y=0.5\pgf@y
            \pgf@x=-\pgf@x
        }
        \anchor{B2}{
            \northwest
            \pgf@y=-0.5\pgf@y
            \pgf@x=-\pgf@x
        }
    }
}

\long\def\pgfrfdeclarequadpole#1#2{
    \pgfrfdeclarenode{#1}{#2}{
        \anchor{A1}{
            \northwest
            \pgf@y=0.5\pgf@y
        }
        \anchor{A2}{
            \northwest
            \pgf@y=-0.5\pgf@y
        }
        \anchor{B1}{
            \northwest
            \pgf@y=0.5\pgf@y
            \pgf@x=-\pgf@x
        }
        \anchor{B2}{
            \northwest
            \pgf@y=-0.5\pgf@y
            \pgf@x=-\pgf@x
        }
    }
}

\pgfrfdeclarebipole{attenuator}{
    \pgfscope             
        \pgftransformscale{.5}
        \pgfnode{resistorshape}{center}{}{pgf@att}{\pgfusepath{stroke}}
    \endpgfscope
    \pgfpathmoveto{\pgfpoint{\pgf@circ@res@left}{0}}
    \pgfpathlineto{\pgfpointanchor{pgf@att}{b}}
    \pgfpathmoveto{\pgfpoint{\pgf@circ@res@right}{0}}
    \pgfpathlineto{\pgfpointanchor{pgf@att}{a}}
    \pgfusepath{draw}

}

\pgfrfdeclarebipole{vattenuator}{
    \pgfscope             
        \pgftransformscale{.5}
        \pgfnode{resistorshape}{center}{}{pgf@att}{\pgfusepath{stroke}}
    \endpgfscope
    \pgfpathmoveto{\pgfpoint{\pgf@circ@res@left}{0}}
    \pgfpathlineto{\pgfpointanchor{pgf@att}{b}}
    \pgfpathmoveto{\pgfpoint{\pgf@circ@res@right}{0}}
    \pgfpathlineto{\pgfpointanchor{pgf@att}{a}}
    \pgfusepath{draw}
    \pgfpathmoveto{\pgfpoint{0.5\pgf@circ@res@left}{0.7\pgf@circ@res@down}}
    \pgfpathlineto{\pgfpoint{0.5\pgf@circ@res@right}{0.7\pgf@circ@res@up}}
    \pgfsetarrowsend{stealth}
    \pgfusepath{draw}
}

\def\pgfrffiltersinewave{
        \pgfpathmoveto{\pgfpoint{\pgf@xb}{\pgf@yb}}
        \pgfpathsine{\pgfpoint{\pgf@xa}{\pgf@ya}}
        \pgfpathcosine{\pgfpoint{\pgf@xa}{-\pgf@ya}}
        \pgfpathsine{\pgfpoint{\pgf@xa}{-\pgf@ya}}
        \pgfpathcosine{\pgfpoint{\pgf@xa}{\pgf@ya}}
        \advance \pgf@yb by \pgf@circ@res@step
}

\long\def\pgfrfdeclarefilter#1#2{
    \pgfrfdeclarebipole{#1}{
        \pgfscope
            \pgf@yb = 0.5\pgf@circ@res@up
            \pgf@circ@res@step = -0.5\pgf@circ@res@up
            
            \pgf@xa = \ctikzvalof{rf/filter/sinewidth}\pgf@circ@res@right
            \divide \pgf@xa by 2
            \pgf@ya = \ctikzvalof{rf/filter/sineheight}\pgf@circ@res@up
            \pgf@xb = \ctikzvalof{rf/filter/sinewidth}\pgf@circ@res@left

            \pgfrffiltersinewave
            \pgfrffiltersinewave
            \pgfrffiltersinewave
            \pgfstroke
        \endpgfscope
        #2
    }
}

\def\pgfrfstrikewave#1{
    \pgf@ya = 0.5\pgf@circ@res@up
    \pgf@yb = #1\pgf@ya
    \advance \pgf@ya by -\pgf@yb
    \advance \pgf@ya by -2pt
    \pgfpathmoveto{\pgfpoint{-2pt}{\pgf@ya}}
    \advance \pgf@ya by 4pt
    \pgfpathlineto{\pgfpoint{2pt}{\pgf@ya}}
}

\pgfrfdeclarefilter{lowpass}{
    \pgfrfstrikewave{0}
    \pgfrfstrikewave{1}
    \pgfusepath{draw}
}

\pgfrfdeclarefilter{highpass}{
    \pgfrfstrikewave{1}
    \pgfrfstrikewave{2}
    \pgfusepath{draw}
}

\pgfrfdeclarefilter{bandpass}{
    \pgfrfstrikewave{0}
    \pgfrfstrikewave{2}
    \pgfusepath{draw}
}

\pgfrfdeclarefilter{allpass}{
}

\pgfrfdeclaretripole{altswitch}{
    \pgf@xa = \ctikzvalof{rf/switch/length}\pgf@circ@res@left
    \pgfpathmoveto{\pgfpoint{\pgf@circ@res@left}{0}}
    \pgfpathlineto{\pgfpoint{\pgf@xa}{0}}
    \pgfpathlineto{\pgfpoint{-\pgf@xa}{-0.5\pgf@circ@res@up}}
    \pgfpathlineto{\pgfpoint{\pgf@circ@res@right}{-0.5\pgf@circ@res@up}}
    \pgfpathmoveto{\pgfpoint{\pgf@circ@res@right}{0.5\pgf@circ@res@up}}
    \pgfpathlineto{\pgfpoint{-\pgf@xa}{0.5\pgf@circ@res@up}}
    \pgfusepath{draw}
    \pgfpathcircle{\pgfpoint{-\ctikzvalof{rf/switch/length}\pgf@circ@res@left}{-0.5\pgf@circ@res@up}}{1.5pt}
    \pgfusepath{draw,fill}
    \pgfsetfillcolor{white}
    \pgfpathcircle{\pgfpoint{-\ctikzvalof{rf/switch/length}\pgf@circ@res@left}{0.5\pgf@circ@res@up}}{1.5pt}
    \pgfusepath{draw,fill}
    \pgfpathmoveto{\pgfpoint{0}{0.25\pgf@circ@res@down}}
    \pgfmathsetmacro{\tikz@start@angle@temp}{(atan2(0.5*\the\pgf@circ@res@up,2*\ctikzvalof{rf/switch/length}*\the\pgf@circ@res@right))}
    \pgfpatharc{-\tikz@start@angle@temp}{+\tikz@start@angle@temp}{\ctikzvalof{rf/switch/length}\pgf@circ@res@right}
    \pgfsetarrowsend{stealth}
    \pgfusepath{draw}
}

\pgfrfdeclarequadpole{chswitch}{
    \pgf@xa = \ctikzvalof{rf/switch/length}\pgf@circ@res@left
    \pgfpathmoveto{\pgfpoint{\pgf@circ@res@left}{0.5\pgf@circ@res@up}}
    \pgfpathlineto{\pgfpoint{\pgf@circ@res@right}{0.5\pgf@circ@res@up}}
    \pgfpathmoveto{\pgfpoint{\pgf@circ@res@left}{0.5\pgf@circ@res@down}}
    \pgfpathlineto{\pgfpoint{\pgf@circ@res@right}{0.5\pgf@circ@res@down}}
    \pgfusepath{draw}
    \pgfpathmoveto{\pgfpoint{\ctikzvalof{rf/switch/length}\pgf@circ@res@left}{0.5\pgf@circ@res@up}}{1.5pt}
    \pgfpathlineto{\pgfpoint{\ctikzvalof{rf/switch/length}\pgf@circ@res@right}{0.5\pgf@circ@res@down}}{1.5pt}
    \pgfpathmoveto{\pgfpoint{\ctikzvalof{rf/switch/length}\pgf@circ@res@right}{0.5\pgf@circ@res@up}}{1.5pt}
    \pgfpathlineto{\pgfpoint{\ctikzvalof{rf/switch/length}\pgf@circ@res@left}{0.5\pgf@circ@res@down}}{1.5pt}
    \pgfsetdash{{3pt}{2pt}}{0pt}
    \pgfusepath{draw}
    \pgfsetfillcolor{white}
    \pgfsetdash{}{0pt}
    \pgfpathcircle{\pgfpoint{\ctikzvalof{rf/switch/length}\pgf@circ@res@left}{0.5\pgf@circ@res@up}}{1.5pt}
    \pgfpathcircle{\pgfpoint{\ctikzvalof{rf/switch/length}\pgf@circ@res@left}{0.5\pgf@circ@res@down}}{1.5pt}
    \pgfpathcircle{\pgfpoint{\ctikzvalof{rf/switch/length}\pgf@circ@res@right}{0.5\pgf@circ@res@up}}{1.5pt}
    \pgfpathcircle{\pgfpoint{\ctikzvalof{rf/switch/length}\pgf@circ@res@right}{0.5\pgf@circ@res@down}}{1.5pt}
    \pgfusepath{draw,fill}
}

\pgfrfdeclarebipoleslash{adc}{
    \pgfscope
        \pgftransformshift{\pgfpoint{0.5\pgf@circ@res@left}{0.5\pgf@circ@res@up}}
        \pgftext{A}
    \endpgfscope
    \pgfscope
        \pgftransformshift{\pgfpoint{0.5\pgf@circ@res@right}{0.5\pgf@circ@res@down}}
        \pgftext{D}
    \endpgfscope
}

\pgfrfdeclarebipoleslash{dac}{
    \pgfscope
        \pgftransformshift{\pgfpoint{0.5\pgf@circ@res@left}{0.5\pgf@circ@res@up}}
        \pgftext{D}
    \endpgfscope
    \pgfscope
        \pgftransformshift{\pgfpoint{0.5\pgf@circ@res@right}{0.5\pgf@circ@res@down}}
        \pgftext{A}
    \endpgfscope
}

\pgfrfdeclarebipole{amp}{
    \pgfscope             
        \pgftransformscale{.5}
        \pgfnode{buffer}{center}{}{pgf@amp}{\pgfusepath{stroke}}
    \endpgfscope
    \pgfpathmoveto{\pgfpoint{\pgf@circ@res@left}{0}}
    \pgfpathlineto{\pgfpointanchor{pgf@amp}{in}}
    \pgfpathmoveto{\pgfpoint{\pgf@circ@res@right}{0}}
    \pgfpathlineto{\pgfpointanchor{pgf@amp}{out}}
    \pgfusepath{draw}
}

\pgfrfdeclaresimplenode{mixer}
{
    \pgf@y=\pgfkeysvalueof{/tikz/circuitikz/bipoles/length}
    \pgf@y=\pgfkeysvalueof{/tikz/circuitikz/tripoles/mixer/height}\pgf@y
    \pgf@y=.4\pgf@y
    \pgf@y=.5\pgf@y
    \pgf@x=\pgfkeysvalueof{/tikz/circuitikz/bipoles/length}
    \pgf@x=.5\pgf@x
    \pgf@x=.4\pgf@x
    \pgf@x=-\pgfkeysvalueof{/tikz/circuitikz/tripoles/mixer/width}\pgf@x
}
{
    \anchorborder{
        \@tempdima=\pgf@x
        \@tempdimb=\pgf@y
        \northwest
        \pgf@xa=-\pgf@x
        \pgf@ya=\pgf@y
        \pgfpointborderellipse{\pgfpoint{\@tempdima}{\@tempdimb}}{\pgfpoint{\pgf@xa}{\pgf@ya}}
    }
}
{
    \northwest
    \pgf@circ@res@up = \pgf@y 
    \pgf@circ@res@down = -\pgf@y
    \pgf@circ@res@right = -\pgf@x
    \pgf@circ@res@left = \pgf@x
    \pgfscope
        \pgfsetlinewidth{\pgfkeysvalueof{/tikz/circuitikz/bipoles/thickness}\pgflinewidth}
        \pgfpathcircle{\pgfpoint{0}{0}}{\pgf@circ@res@up}
        \pgfusepath{draw}
    \endpgfscope
    \pgfmathsetlength{\pgf@circ@res@step}{\the\pgf@circ@res@right*cos(45)}
    \pgfpathmoveto{\pgfpoint{\pgf@circ@res@step}{\pgf@circ@res@step}}
    \pgfpathlineto{\pgfpoint{-\pgf@circ@res@step}{-\pgf@circ@res@step}}
    \pgfpathmoveto{\pgfpoint{-\pgf@circ@res@step}{\pgf@circ@res@step}}
    \pgfpathlineto{\pgfpoint{\pgf@circ@res@step}{-\pgf@circ@res@step}}
    \pgfusepath{draw}
}


\pgfrfdeclarenode{iqmix}{
    \pgfscope             
        \pgftransformshift{\pgfpoint{0.5\pgf@circ@res@right}{0.5\pgf@circ@res@up}}
        \pgftransformrotate{180}
        \pgftransformscale{.2}
        \pgfnode{mixer}{center}{}{pgf@mixi}{\pgfusepath{stroke}}
    \endpgfscope
    \pgfscope             
        \pgftransformshift{\pgfpoint{0}{0.5\pgf@circ@res@down}}
        \pgftransformrotate{180}
        \pgftransformscale{.2}
        \pgfnode{mixer}{center}{}{pgf@mixq}{\pgfusepath{stroke}}
    \endpgfscope
    \pgfscope             
        \pgftransformscale{.35}
        \pgfnode{rectangle}{center}{$90\degree$}{pgf@phase}{\pgfusepath{stroke}}
    \endpgfscope
    \pgfpathmoveto{\pgfpoint{\pgf@circ@res@left}{0pt}}
    \pgfpathlineto{\pgfpoint{0.6\pgf@circ@res@left}{0pt}}
    \pgfpathlineto{\pgfpoint{0.6\pgf@circ@res@left}{0.5\pgf@circ@res@up}}
    \pgfpathlineto{\pgfpointanchor{pgf@mixi}{out}}
    \pgfpathmoveto{\pgfpoint{0.6\pgf@circ@res@left}{0pt}}
    \pgfpathlineto{\pgfpoint{0.6\pgf@circ@res@left}{0.5\pgf@circ@res@down}}
    \pgfpathlineto{\pgfpointanchor{pgf@mixq}{out}}
    \pgfpathmoveto{\pgfpointanchor{pgf@mixi}{in}}
    \pgfpathlineto{\pgfpoint{\pgf@circ@res@right}{0.5\pgf@circ@res@up}}
    \pgfpathmoveto{\pgfpointanchor{pgf@mixq}{in}}
    \pgfpathlineto{\pgfpoint{\pgf@circ@res@right}{0.5\pgf@circ@res@down}}
    \pgfpathmoveto{\pgfpoint{0pt}{\pgf@circ@res@up}}
    \pgfpathlineto{\pgfpointanchor{pgf@phase}{north}}
    \pgfpathmoveto{\pgfpointanchor{pgf@phase}{south}}
    \pgfpathlineto{\pgfpointanchor{pgf@mixq}{in 2}}
    \pgfpathmoveto{\pgfpoint{0pt}{0.8\pgf@circ@res@up}}
    \pgfpathlineto{\pgfpoint{0.5\pgf@circ@res@right}{0.8\pgf@circ@res@up}}
    \pgfpathlineto{\pgfpointanchor{pgf@mixi}{in 2}}
    \pgfusepath{draw}
    \pgfpathcircle{\pgfpoint{0pt}{0.8\pgf@circ@res@up}}{1pt}
    \pgfpathcircle{\pgfpoint{0.6\pgf@circ@res@left}{0pt}}{1pt}
    \pgfusepath{fill}

}{
    \anchor{A1}{
        \northwest
        \pgf@y=0pt
    }
    \anchor{B1}{
        \northwest
        \pgf@y=0.5\pgf@y
        \pgf@x=-\pgf@x
    }
    \anchor{B2}{
        \northwest
        \pgf@y=-0.5\pgf@y
        \pgf@x=-\pgf@x
    }
    \anchor{C1}{
        \northwest
        \pgf@x=0pt
    }
}

\pgfrfdeclarenode{iqmixdown}{
    \pgfscope             
        \pgftransformshift{\pgfpoint{0.5\pgf@circ@res@right}{0.5\pgf@circ@res@down}}
        \pgftransformscale{.2}
        \pgfnode{mixer}{center}{}{pgf@mixi}{\pgfusepath{stroke}}
    \endpgfscope
    \pgfscope             
        \pgftransformshift{\pgfpoint{0}{0.5\pgf@circ@res@up}}
        \pgftransformscale{.2}
        \pgfnode{mixer}{center}{}{pgf@mixq}{\pgfusepath{stroke}}
    \endpgfscope
    \pgfscope             
        \pgftransformscale{.35}
        \pgfnode{rectangle}{center}{$90\degree$}{pgf@phase}{\pgfusepath{stroke}}
    \endpgfscope
    \pgfpathmoveto{\pgfpoint{\pgf@circ@res@left}{0pt}}
    \pgfpathlineto{\pgfpoint{0.6\pgf@circ@res@left}{0pt}}
    \pgfpathlineto{\pgfpoint{0.6\pgf@circ@res@left}{0.5\pgf@circ@res@down}}
    \pgfpathlineto{\pgfpointanchor{pgf@mixi}{in}}
    \pgfpathmoveto{\pgfpoint{0.6\pgf@circ@res@left}{0pt}}
    \pgfpathlineto{\pgfpoint{0.6\pgf@circ@res@left}{0.5\pgf@circ@res@up}}
    \pgfpathlineto{\pgfpointanchor{pgf@mixq}{in}}
    \pgfpathmoveto{\pgfpointanchor{pgf@mixi}{out}}
    \pgfpathlineto{\pgfpoint{\pgf@circ@res@right}{0.5\pgf@circ@res@down}}
    \pgfpathmoveto{\pgfpointanchor{pgf@mixq}{out}}
    \pgfpathlineto{\pgfpoint{\pgf@circ@res@right}{0.5\pgf@circ@res@up}}
    \pgfpathmoveto{\pgfpoint{0pt}{\pgf@circ@res@down}}
    \pgfpathlineto{\pgfpointanchor{pgf@phase}{south}}
    \pgfpathmoveto{\pgfpointanchor{pgf@phase}{north}}
    \pgfpathlineto{\pgfpointanchor{pgf@mixq}{in 2}}
    \pgfpathmoveto{\pgfpoint{0pt}{0.8\pgf@circ@res@down}}
    \pgfpathlineto{\pgfpoint{0.5\pgf@circ@res@right}{0.8\pgf@circ@res@down}}
    \pgfpathlineto{\pgfpointanchor{pgf@mixi}{in 2}}
    \pgfusepath{draw}
    \pgfpathcircle{\pgfpoint{0pt}{0.8\pgf@circ@res@down}}{1pt}
    \pgfpathcircle{\pgfpoint{0.6\pgf@circ@res@left}{0pt}}{1pt}
    \pgfusepath{fill}

}{
    \anchor{A1}{
        \northwest
        \pgf@y=0pt
    }
    \anchor{B1}{
        \northwest
        \pgf@y=0.5\pgf@y
        \pgf@x=-\pgf@x
    }
    \anchor{B2}{
        \northwest
        \pgf@y=-0.5\pgf@y
        \pgf@x=-\pgf@x
    }
    \anchor{C1}{
        \northwest
        \pgf@x=0pt
        \pgf@y=-\pgf@y
    }
}

\pgfrfdeclarebipole{pll}{
    \pgftext{PLL}
}

\pgfrfdeclarebipole{dut}{
    \pgftext{DUT}
}

\pgfrfdeclarebipole{vna}{
    \pgftext{VNA}
}

\pgfrfdeclarebipole{empty}{
}

\pgfrfdeclaresimplenode{source}
{
    \pgf@y=\pgfkeysvalueof{/tikz/circuitikz/bipoles/length}
    \pgf@y=\pgfkeysvalueof{/tikz/circuitikz/bipoles/vsource/height}\pgf@y
    \pgf@y=.5\pgf@y
    \pgf@x=\pgfkeysvalueof{/tikz/circuitikz/bipoles/length}
    \pgf@x=.5\pgf@x
    \pgf@x=-\pgfkeysvalueof{/tikz/circuitikz/bipoles/vsource/width}\pgf@x
}
{
    \anchorborder{
        \@tempdima=\pgf@x
        \@tempdimb=\pgf@y
        \northwest
        \pgf@xa=-\pgf@x
        \pgf@ya=\pgf@y
        \pgfpointborderellipse{\pgfpoint{\@tempdima}{\@tempdimb}}{\pgfpoint{\pgf@xa}{\pgf@ya}}
    }
}
{
    \northwest
    \pgf@circ@res@up = \pgf@y 
    \pgf@circ@res@down = -\pgf@y
    \pgf@circ@res@right = -\pgf@x
    \pgf@circ@res@left = \pgf@x
    \pgfscope             
        \pgftransformrotate{90}
        \pgfnode{vsourcesinshape}{center}{}{pgf@att}{\pgfusepath{stroke}}
    \endpgfscope
}

\pgfrfdeclaresimplenode{isolator}
{
    \pgf@y=\pgfkeysvalueof{/tikz/circuitikz/bipoles/length}
    \pgf@y=\pgfkeysvalueof{/tikz/circuitikz/bipoles/vsource/height}\pgf@y
    \pgf@y=.5\pgf@y
    \pgf@x=\pgfkeysvalueof{/tikz/circuitikz/bipoles/length}
    \pgf@x=.5\pgf@x
    \pgf@x=-\pgfkeysvalueof{/tikz/circuitikz/bipoles/vsource/width}\pgf@x
}
{
    \anchorborder{
        \@tempdima=\pgf@x
        \@tempdimb=\pgf@y
        \northwest
        \pgf@xa=-\pgf@x
        \pgf@ya=\pgf@y
        \pgfpointborderellipse{\pgfpoint{\@tempdima}{\@tempdimb}}{\pgfpoint{\pgf@xa}{\pgf@ya}}
    }
}
{
    \northwest
    \pgf@circ@res@up = \pgf@y 
    \pgf@circ@res@down = -\pgf@y
    \pgf@circ@res@right = -\pgf@x
    \pgf@circ@res@left = \pgf@x
    \pgfscope
        \pgfsetlinewidth{\pgfkeysvalueof{/tikz/circuitikz/bipoles/thickness}\pgflinewidth}
        \pgfpathcircle{\pgfpoint{0}{0}}{\pgf@circ@res@up}
        \pgfusepath{draw}
        \pgfsetarrowsend{latex}
        \pgfpathmoveto{\pgfpoint{0.8\pgf@circ@res@left}{0pt}}
        \pgfpathlineto{\pgfpoint{0.8\pgf@circ@res@right}{0pt}}
        \pgfusepath{draw}
    \endpgfscope
}

\pgfrfdeclaresimplenode{vphase}
{
    \pgf@y=\pgfkeysvalueof{/tikz/circuitikz/bipoles/length}
    \pgf@y=\pgfkeysvalueof{/tikz/circuitikz/bipoles/vsource/height}\pgf@y
    \pgf@y=.5\pgf@y
    \pgf@x=\pgfkeysvalueof{/tikz/circuitikz/bipoles/length}
    \pgf@x=.5\pgf@x
    \pgf@x=-\pgfkeysvalueof{/tikz/circuitikz/bipoles/vsource/width}\pgf@x
}
{
    \anchorborder{
        \@tempdima=\pgf@x
        \@tempdimb=\pgf@y
        \northwest
        \pgf@xa=-\pgf@x
        \pgf@ya=\pgf@y
        \pgfpointborderellipse{\pgfpoint{\@tempdima}{\@tempdimb}}{\pgfpoint{\pgf@xa}{\pgf@ya}}
    }
}
{
    \northwest
    \pgf@circ@res@up = \pgf@y 
    \pgf@circ@res@down = -\pgf@y
    \pgf@circ@res@right = -\pgf@x
    \pgf@circ@res@left = \pgf@x
    \pgfscope
        \pgfsetlinewidth{\pgfkeysvalueof{/tikz/circuitikz/bipoles/thickness}\pgflinewidth}
        \pgfpathcircle{\pgfpoint{0}{0}}{\pgf@circ@res@up}
        \pgfusepath{draw}
    \endpgfscope
    \pgfscope
        \pgfsetarrowsend{latex}
        \pgfpathmoveto{\pgfpoint{\pgf@circ@res@right}{\pgf@circ@res@up}}
        \pgfpathlineto{\pgfpoint{\pgf@circ@res@left}{\pgf@circ@res@down}}
        \pgfusepath{draw}
    \endpgfscope
}

\pgfrfdeclaresimplenode{tuner}
{
    \pgf@y=\pgfkeysvalueof{/tikz/circuitikz/bipoles/length}
    \pgf@y=\pgfkeysvalueof{/tikz/circuitikz/bipoles/tgeneric/height}\pgf@y
    \pgf@y=.5\pgf@y
    \pgf@x=\pgfkeysvalueof{/tikz/circuitikz/bipoles/length}
    \pgf@x=.5\pgf@x
    \pgf@x=-\pgfkeysvalueof{/tikz/circuitikz/bipoles/tgeneric/width}\pgf@x
}
{
    \anchorborder{
        \@tempdima=\pgf@x
        \@tempdimb=\pgf@y
        \northwest
        \pgf@xa=-\pgf@x
        \pgf@ya=\pgf@y
        \pgfpointborderrectangle{\pgfpoint{\@tempdima}{\@tempdimb}}{\pgfpoint{\pgf@xa}{\pgf@ya}}
    }
}
{
    \northwest
    \pgf@circ@res@up = \pgf@y 
    \pgf@circ@res@down = -\pgf@y
    \pgf@circ@res@right = -\pgf@x
    \pgf@circ@res@left = \pgf@x
    \pgfscope             
        \pgfnode{tgenericshape}{center}{}{pgf@att}{\pgfusepath{stroke}}
    \endpgfscope
}

\pgfrfdeclaresimplenode{amplifier}
{
    \pgf@y=\pgfkeysvalueof{/tikz/circuitikz/bipoles/length}
    \pgf@y=\pgfkeysvalueof{/tikz/circuitikz/bipoles/buffer/height}\pgf@y
    \pgf@y=.5\pgf@y
    \pgf@y=.6\pgf@y
    \pgf@x=\pgfkeysvalueof{/tikz/circuitikz/bipoles/length}
    \pgf@x=.5\pgf@x
    \pgf@x=.4\pgf@x
    \pgf@x=-\pgfkeysvalueof{/tikz/circuitikz/bipoles/buffer/width}\pgf@x
}
{
    \anchorborder{
        \@tempdima=\pgf@x
        \@tempdimb=\pgf@y
        \northwest
        \pgf@xa=-\pgf@x
        \pgf@ya=\pgf@y
        \pgfpointborderrectangle{\pgfpoint{\@tempdima}{\@tempdimb}}{\pgfpoint{\pgf@xa}{\pgf@ya}}
    }
}
{
    \northwest
    \pgf@circ@res@up = \pgf@y 
    \pgf@circ@res@down = -\pgf@y
    \pgf@circ@res@right = -\pgf@x
    \pgf@circ@res@left = \pgf@x
    \pgfscope
        \pgfsetlinewidth{\pgfkeysvalueof{/tikz/circuitikz/bipoles/thickness}\pgflinewidth}
        \pgfpathmoveto{\pgfpoint{\pgf@circ@res@left}{0pt}}
        \pgfpathlineto{\pgfpoint{\pgf@circ@res@right}{\pgf@circ@res@up}}
        \pgfpathlineto{\pgfpoint{\pgf@circ@res@right}{\pgf@circ@res@down}}
        \pgfpathclose
        \pgfusepath{stroke}
    \endpgfscope
}

\pgfrfdeclaresimplenode{match}
{
    \pgf@y=\pgfkeysvalueof{/tikz/circuitikz/bipoles/length}
    \pgf@y=\pgfkeysvalueof{/tikz/circuitikz/bipoles/resistor/height}\pgf@y
    \pgf@y=.5\pgf@y
    \pgf@x=\pgfkeysvalueof{/tikz/circuitikz/bipoles/length}
    \pgf@x=.5\pgf@x
    \pgf@x=-\pgfkeysvalueof{/tikz/circuitikz/bipoles/resistor/width}\pgf@x
}
{
    \anchorborder{
        \@tempdima=\pgf@x
        \@tempdimb=\pgf@y
        \northwest
        \pgf@xa=-\pgf@x
        \pgf@ya=\pgf@y
        \pgfpointborderrectangle{\pgfpoint{\@tempdima}{\@tempdimb}}{\pgfpoint{\pgf@xa}{\pgf@ya}}
    }
}
{
    \northwest
    \pgf@circ@res@up = \pgf@y 
    \pgf@circ@res@down = -\pgf@y
    \pgf@circ@res@right = -\pgf@x
    \pgf@circ@res@left = \pgf@x
    \pgfscope             
        \pgfnode{resistorshape}{center}{}{pgf@att}{\pgfusepath{stroke}}
    \endpgfscope
}

\pgfrfdeclaresimplenode{vmatch}
{
    \pgf@y=\pgfkeysvalueof{/tikz/circuitikz/bipoles/length}
    \pgf@y=\pgfkeysvalueof{/tikz/circuitikz/bipoles/vresistor/height}\pgf@y
    \pgf@y=.5\pgf@y
    \pgf@x=\pgfkeysvalueof{/tikz/circuitikz/bipoles/length}
    \pgf@x=.5\pgf@x
    \pgf@x=-\pgfkeysvalueof{/tikz/circuitikz/bipoles/vresistor/width}\pgf@x
}
{
    \anchorborder{
        \@tempdima=\pgf@x
        \@tempdimb=\pgf@y
        \northwest
        \pgf@xa=-\pgf@x
        \pgf@ya=\pgf@y
        \pgfpointborderrectangle{\pgfpoint{\@tempdima}{\@tempdimb}}{\pgfpoint{\pgf@xa}{\pgf@ya}}
    }
}
{
    \northwest
    \pgf@circ@res@up = \pgf@y 
    \pgf@circ@res@down = -\pgf@y
    \pgf@circ@res@right = -\pgf@x
    \pgf@circ@res@left = \pgf@x
    \pgfscope             
        \pgfnode{vresistorshape}{center}{}{pgf@att}{\pgfusepath{stroke}}
    \endpgfscope
}

\long\def\pgfrfdeclarecoupler#1#2{
    \pgfrfdeclaredoublenode{#1}{
        \pgfpathmoveto{\pgfpoint{\pgf@circ@res@left}{0pt}}
        \pgfpathlineto{\pgfpoint{\pgf@circ@res@right}{0pt}}
        \pgfusepath{draw}
        \pgf@xa = 0.75\pgf@circ@res@left
        \pgfscope
            \pgfsetlinewidth{1pt}
            \advance \pgf@xa by 4pt
            \pgfpathmoveto{\pgfpoint{\pgf@xa}{0pt}}
            \pgfpathlineto{\pgfpoint{-\pgf@xa}{0pt}}
            \pgfpathmoveto{\pgfpoint{\pgf@xa}{0.15\pgf@circ@res@up}}
            \pgfpathlineto{\pgfpoint{-\pgf@xa}{0.15\pgf@circ@res@up}}
            \pgfusepath{draw}
        \endpgfscope
        \pgfscope
            \pgfsetcornersarced{\pgfpoint{4pt}{4pt}}
            #2
            \pgfusepath{draw}
        \endpgfscope
    }{
        \anchor{A1}{
            \northwest
            \pgf@y=0pt
        }
        \anchor{A2}{
            \northwest
            \pgf@x=0.75\pgf@x
        }
        \anchor{B1}{
            \northwest
            \pgf@x=-\pgf@x
            \pgf@y=0pt
        }
        \anchor{B2}{
            \northwest
            \pgf@x=-0.75\pgf@x
        }
    }
}

\pgfrfdeclarecoupler{dircoupler}{
    \pgfpathmoveto{\pgfpoint{0.75\pgf@circ@res@left}{\pgf@circ@res@up}}
    \pgfpathlineto{\pgfpoint{0.75\pgf@circ@res@left}{0.15\pgf@circ@res@up}}
    \pgfpathlineto{\pgfpoint{0.75\pgf@circ@res@right}{0.15\pgf@circ@res@up}}
    \pgfpathlineto{\pgfpoint{0.75\pgf@circ@res@right}{\pgf@circ@res@up}}
}

\pgfrfdeclarecoupler{dircouplera}{
    \pgfpathmoveto{\pgfpoint{0.75\pgf@circ@res@left}{\pgf@circ@res@up}}
    \pgfpathlineto{\pgfpoint{0.75\pgf@circ@res@left}{0.15\pgf@circ@res@up}}
    \pgfpathlineto{\pgfpoint{0pt}{0.15\pgf@circ@res@up}}
}

\pgfrfdeclarecoupler{dircouplerb}{
    \pgfpathmoveto{\pgfpoint{0.75\pgf@circ@res@left}{0.15\pgf@circ@res@up}}
    \pgfpathlineto{\pgfpoint{0.75\pgf@circ@res@right}{0.15\pgf@circ@res@up}}
    \pgfpathlineto{\pgfpoint{0pt}{0.15\pgf@circ@res@up}}
}

\pgfrfdeclaremonopole{generator}{
    \pgfscope             
        \pgftransformscale{.5}
        \pgftransformrotate{90}
        \pgfnode{vsourcesinshape}{center}{}{pgf@att}{\pgfusepath{stroke}}
    \endpgfscope
}

\pgfrfdeclaremonopole{powermeter}{
    \pgfscope             
        \pgftransformscale{.5}
        \pgftransformrotate{-90}
        \pgfnode{fulldiodeshape}{center}{}{pgf@att}{\pgfusepath{stroke}}
    \endpgfscope
    \pgfpathmoveto{\pgfpoint{0pt}{0.7\pgf@circ@res@up}}
    \pgfpathlineto{\pgfpoint{0pt}{0.7\pgf@circ@res@down}}
    \pgfusepath{draw}
}

\pgfrfdeclaremonopole{spectrum analyzer}{
    \pgfpathmoveto{\pgfpoint{0.7\pgf@circ@res@down}{0.8\pgf@circ@res@left}}
    \pgfpathlineto{\pgfpoint{0.7\pgf@circ@res@down}{0.7\pgf@circ@res@right}}
    \pgfpathmoveto{\pgfpoint{0.8\pgf@circ@res@down}{0.7\pgf@circ@res@left}}
    \pgfpathlineto{\pgfpoint{0.7\pgf@circ@res@up}{0.7\pgf@circ@res@left}}
    \pgfusepath{draw}

    \pgfsetcornersarced{\pgfpoint{2pt}{1pt}}
    \pgfpathmoveto{\pgfpoint{0.7\pgf@circ@res@left}{0.6\pgf@circ@res@down}}
    \pgfpathlineto{\pgfpoint{0.55\pgf@circ@res@left}{0.6\pgf@circ@res@down}}
    \pgfpathlineto{\pgfpoint{0.5\pgf@circ@res@left}{0.4\pgf@circ@res@up}}
    \pgfpathlineto{\pgfpoint{0.45\pgf@circ@res@left}{0.6\pgf@circ@res@down}}
    \pgfpathlineto{\pgfpoint{0.05\pgf@circ@res@left}{0.6\pgf@circ@res@down}}
    \pgfpathlineto{\pgfpoint{0}{0.6\pgf@circ@res@up}}
    \pgfpathlineto{\pgfpoint{0.05\pgf@circ@res@right}{0.6\pgf@circ@res@down}}
    \pgfpathlineto{\pgfpoint{0.45\pgf@circ@res@right}{0.6\pgf@circ@res@down}}
    \pgfpathlineto{\pgfpoint{0.5\pgf@circ@res@right}{0.4\pgf@circ@res@up}}
    \pgfpathlineto{\pgfpoint{0.55\pgf@circ@res@right}{0.6\pgf@circ@res@down}}
    \pgfpathlineto{\pgfpoint{0.7\pgf@circ@res@right}{0.6\pgf@circ@res@down}}
    \pgfusepath{draw}
}

\pgfrfdeclaretripole{splitter}{
    \pgfscope             
        \pgftransformshift{\pgfpoint{0.5\pgf@circ@res@right}{0}}
        \pgftransformscale{.2}
        \pgftransformrotate{-90}
        \pgfnode{resistorshape}{center}{}{pgf@att}{\pgfusepath{stroke}}
    \endpgfscope
    \pgfpathmoveto{\pgfpoint{\pgf@circ@res@left}{0}}
    \pgfpathlineto{\pgfpoint{.5\pgf@circ@res@left}{0}}
    \pgfpathlineto{\pgfpoint{.5\pgf@circ@res@right}{.5\pgf@circ@res@up}}
    \pgfpathlineto{\pgfpoint{\pgf@circ@res@right}{.5\pgf@circ@res@up}}
    \pgfpathmoveto{\pgfpoint{.5\pgf@circ@res@left}{0}}
    \pgfpathlineto{\pgfpoint{.5\pgf@circ@res@right}{.5\pgf@circ@res@down}}
    \pgfpathlineto{\pgfpoint{\pgf@circ@res@right}{.5\pgf@circ@res@down}}
    \pgfpathmoveto{\pgfpointanchor{pgf@att}{east}}
    \pgfpathlineto{\pgfpoint{.5\pgf@circ@res@right}{.5\pgf@circ@res@down}}
    \pgfpathmoveto{\pgfpointanchor{pgf@att}{west}}
    \pgfpathlineto{\pgfpoint{.5\pgf@circ@res@right}{.5\pgf@circ@res@up}}
    \pgfusepath{draw}
}

\pgfrfdeclarebipole{dc block}{
    \pgfscope             
        \pgftransformscale{.5}
        \pgfnode{capacitorshape}{center}{}{pgf@att}{\pgfusepath{stroke}}
    \endpgfscope
    \pgfpathmoveto{\pgfpoint{\pgf@circ@res@left}{0}}
    \pgfpathlineto{\pgfpointanchor{pgf@att}{b}}
    \pgfpathmoveto{\pgfpoint{\pgf@circ@res@right}{0}}
    \pgfpathlineto{\pgfpointanchor{pgf@att}{a}}
    \pgfusepath{draw}

}

\pgfrfdeclaredoublenode{oscilloscope}{
    \pgfscope
        \pgfsetcornersarced{\pgfpoint{2pt}{2pt}}
        \pgfpathrectanglecorners{\pgfpoint{0.9\pgf@circ@res@left}{0.4\pgf@circ@res@down}}{\pgfpoint{0.1\pgf@circ@res@left}{0.75\pgf@circ@res@up}}
        \pgfusepath{draw}
    \endpgfscope
    \foreach \posc/\post/\i in {0.2/0.15/4,0.4/0.35/3,0.6/0.55/2,0.8/0.75/1}{
        \pgfscope
            \pgftransformshift{\pgfpoint{\post\pgf@circ@res@left}{0.6\pgf@circ@res@down}}
            \pgftransformscale{.4}
            \pgftext{\i}
            \pgfusepath{draw}
        \endpgfscope
        \pgfpathcircle{\pgfpoint{\posc\pgf@circ@res@left}{0.8\pgf@circ@res@down}}{1pt}
        \pgfusepath{draw}
    }
    \foreach \x in {0,0.2,0.4}{
        \foreach \y in {0.55,0.25,-0.05,-0.35}{
            \pgfpathrectangle{\pgfpoint{\x\pgf@circ@res@right}{\y\pgf@circ@res@up}}{\pgfpoint{2pt}{2pt}}
        }
    }
    \pgfpathrectangle{\pgfpoint{0.6\pgf@circ@res@right}{0\pgf@circ@res@up}}{\pgfpoint{2pt}{2pt}}
    \pgfpathrectangle{\pgfpoint{0.8\pgf@circ@res@right}{0\pgf@circ@res@up}}{\pgfpoint{2pt}{2pt}}
    \pgfpathcircle{\pgfpoint{0.75\pgf@circ@res@right}{0.45\pgf@circ@res@up}}{0.15\pgf@circ@res@right}
    \pgfpathmoveto{\pgfpoint{0.9\pgf@circ@res@left}{0pt}}
    \pgfpathsine{\pgfpoint{0.2\pgf@circ@res@right}{0.2\pgf@circ@res@up}}
    \pgfpathcosine{\pgfpoint{0.2\pgf@circ@res@right}{-0.2\pgf@circ@res@up}}
    \pgfpathsine{\pgfpoint{0.2\pgf@circ@res@right}{-0.2\pgf@circ@res@up}}
    \pgfpathcosine{\pgfpoint{0.2\pgf@circ@res@right}{0.2\pgf@circ@res@up}}
    \pgfpathmoveto{\pgfpoint{0.9\pgf@circ@res@left}{0.4\pgf@circ@res@up}}
    \pgfpathsine{\pgfpoint{0.1\pgf@circ@res@right}{0.1\pgf@circ@res@up}}
    \pgfpathcosine{\pgfpoint{0.1\pgf@circ@res@right}{-0.1\pgf@circ@res@up}}
    \pgfpathsine{\pgfpoint{0.1\pgf@circ@res@right}{-0.1\pgf@circ@res@up}}
    \pgfpathcosine{\pgfpoint{0.1\pgf@circ@res@right}{0.1\pgf@circ@res@up}}
    \pgfpathsine{\pgfpoint{0.1\pgf@circ@res@right}{0.1\pgf@circ@res@up}}
    \pgfpathcosine{\pgfpoint{0.1\pgf@circ@res@right}{-0.1\pgf@circ@res@up}}
    \pgfpathsine{\pgfpoint{0.1\pgf@circ@res@right}{-0.1\pgf@circ@res@up}}
    \pgfpathcosine{\pgfpoint{0.1\pgf@circ@res@right}{0.1\pgf@circ@res@up}}
    \pgfusepath{draw}

}{
    \anchor{A1}{
        \northwest
        \pgf@y=-0.8\pgf@y
        \pgf@x=0.8\pgf@x
    }
    \anchor{A2}{
        \northwest
        \pgf@y=-0.8\pgf@y
        \pgf@x=0.6\pgf@x
    }
    \anchor{A3}{
        \northwest
        \pgf@y=-0.8\pgf@y
        \pgf@x=0.4\pgf@x
    }
    \anchor{A4}{
        \northwest
        \pgf@y=-0.8\pgf@y
        \pgf@x=0.2\pgf@x
    }
}

\pgfrfdeclaresimplenode{circulator}
{
    \pgf@y=\pgfkeysvalueof{/tikz/circuitikz/bipoles/length}
    \pgf@y=\pgfkeysvalueof{/tikz/circuitikz/tripoles/mixer/height}\pgf@y
    \pgf@y=.5\pgf@y
    \pgf@y=\pgfkeysvalueof{/tikz/circuitikz/tripoles/mixer/margin}\pgf@y
    \pgf@x=\pgfkeysvalueof{/tikz/circuitikz/bipoles/length}
    \pgf@x=.5\pgf@x
    \pgf@x=-\pgfkeysvalueof{/tikz/circuitikz/tripoles/mixer/width}\pgf@x
    \pgf@x=\pgfkeysvalueof{/tikz/circuitikz/tripoles/mixer/margin}\pgf@x
}
{
    \anchor{A}{
        \northwest
        \pgf@y=0pt
    }
    \anchor{B}{
        \northwest
        \pgfpointpolar{45}{\pgf@y}
    }
    \anchor{C}{
        \northwest
        \pgfpointpolar{-45}{\pgf@y}
    }
    \anchorborder{
        \@tempdima=\pgf@x
        \@tempdimb=\pgf@y
        \northwest
        \pgf@xa=-\pgf@x
        \pgf@ya=\pgf@y
        \pgfpointborderellipse{\pgfpoint{\@tempdima}{\@tempdimb}}{\pgfpoint{\pgf@xa}{\pgf@ya}}
    }
}
{
    \northwest
    \pgf@circ@res@up = \pgf@y 
    \pgf@circ@res@down = -\pgf@y
    \pgf@circ@res@right = -\pgf@x
    \pgf@circ@res@left = \pgf@x

    \pgfscope
        \pgfsetlinewidth{\pgfkeysvalueof{/tikz/circuitikz/bipoles/thickness}\pgflinewidth}
        \pgfpathcircle{\pgfpoint{0}{0}}{\pgf@circ@res@up}
        \pgfusepath{draw}
    \endpgfscope
    \pgfscope
        \pgfsetarrows{-{latex[length=5mm]}}
        \pgfpathmoveto{\pgfpoint{-0.65\pgf@circ@res@up}{0}}
        \pgfpatharc{180}{0} {0.65\pgf@circ@res@up}
        \pgfusepath{draw}
    \endpgfscope
}

\pgfdeclareshape{vectorgenerator}
{
    \anchor{center}{
        \pgfpointorigin
    }
    \savedanchor\northwest{%
        \pgf@y=\pgfkeysvalueof{/tikz/circuitikz/bipoles/length}
        \pgf@y=\pgfkeysvalueof{/tikz/circuitikz/rf/height}\pgf@y
        \pgf@y=0.75\pgf@y
        \pgf@x=\pgfkeysvalueof{/tikz/circuitikz/bipoles/length}
        \pgf@x=\pgf@x
        \pgf@x=-\pgfkeysvalueof{/tikz/circuitikz/rf/width}\pgf@x
    }
    \anchor{north}{
        \northwest
        \pgf@x=0pt
    }
    \anchor{south}{
        \northwest
        \pgf@x=0pt
        \pgf@y=-\pgf@y
    }
    \anchor{out}{
        \northwest
        \pgf@y=0.33\pgf@y
    }
    \anchor{in}{
        \northwest
        \pgf@x=-\pgf@x
        \pgf@y=-0.33\pgf@y
    }
    \anchor{west}{
        \northwest
        \pgf@y=0pt
    }
    \anchor{east}{
        \northwest
        \pgf@y=0pt
        \pgf@x=-\pgf@x
    }
    \anchor{south west}{
        \northwest
        \pgf@y=-\pgf@y
    }
    \anchor{north east}{
        \northwest
        \pgf@x=-\pgf@x
    }
    \anchor{north west}{
        \northwest
    }
    \anchor{south east}{
        \northwest
        \pgf@x=-\pgf@x
        \pgf@y=-\pgf@y
    }	  
    \anchor{base}{
        \northwest
        \pgf@x=0pt	  	
    }
    \anchorborder{
        \@tempdima=\pgf@x
        \@tempdimb=\pgf@y
        \northwest
        \pgf@xa=-\pgf@x
        \pgf@ya=\pgf@y
        \pgfpointborderrectangle{\pgfpoint{\@tempdima}{\@tempdimb}}{\pgfpoint{\pgf@xa}{\pgf@ya}}
    }
    \backgroundpath{			
        \pgfsetcolor{\pgfkeysvalueof{/tikz/circuitikz/color}}	

        \northwest
        \pgf@circ@res@up = \pgf@y 
        \pgf@circ@res@down = -\pgf@y
        \pgf@circ@res@right = -\pgf@x
        \pgf@circ@res@left = \pgf@x
        \pgf@circ@res@other = 0.33\pgf@circ@res@up

        \pgfscope
            \pgfsetcornersarced{\pgfpoint{4pt}{4pt}}
            \pgfpathrectanglecorners{\pgfpoint{\pgf@circ@res@left}{\pgf@circ@res@down}}{\pgfpoint{\pgf@circ@res@right}{\pgf@circ@res@up}}
            \pgfstroke
        \endpgfscope
        \pgfscope             
            \pgftransformshift{\pgfpoint{0.66\pgf@circ@res@left}{\pgf@circ@res@other}}
            \pgftransformscale{.4}
            \pgftransformrotate{180}
            \pgfnode{buffer}{center}{}{pgf@amp}{\pgfusepath{stroke}}
        \endpgfscope
        \pgfscope
            \pgftransformshift{\pgfpoint{0}{\pgf@circ@res@other}}
            \pgftransformscale{.5}
            \pgfnode{mixer}{center}{}{pgf@mix}{\pgfusepath{stroke}}
        \endpgfscope
        \pgfscope
            \pgftransformshift{\pgfpoint{0.66\pgf@circ@res@right}{\pgf@circ@res@other}}
            \pgftransformscale{.5}
            \pgftransformrotate{90}
            \pgfnode{vsourcesinshape}{center}{}{pgf@att}{\pgfusepath{stroke}}
        \endpgfscope
        \pgfscope
            \pgfpathmoveto{\pgfpoint{0.77\pgf@circ@res@left}{0.17\pgf@circ@res@down}}
            \pgfpathlineto{\pgfpoint{0.55\pgf@circ@res@left}{0.83\pgf@circ@res@up}}
            \pgfsetarrowsend{stealth}
            \pgfusepath{draw}
        \endpgfscope
        \pgfscope
            \pgfsetlinewidth{\pgfkeysvalueof{/tikz/circuitikz/bipoles/thickness}\pgflinewidth}
            \pgftransformshift{\pgfpoint{0.66\pgf@circ@res@right}{-\pgf@circ@res@other}}
            \pgftransformscale{.5}
            \pgfnode{adc}{center}{}{pgf@dac}{\pgfusepath{stroke}}
        \endpgfscope
        \pgfpathmoveto{\pgfpoint{\pgf@circ@res@left}{\pgf@circ@res@other}}
        \pgfpathlineto{\pgfpointanchor{pgf@amp}{out}}
        \pgfpathmoveto{\pgfpointanchor{pgf@amp}{in}}
        \pgfpathlineto{\pgfpointanchor{pgf@mix}{in}}
        \pgfpathmoveto{\pgfpointanchor{pgf@mix}{out}}
        \pgfpathlineto{\pgfpointanchor{pgf@att}{north}}
        \pgfpathmoveto{\pgfpointanchor{pgf@mix}{in 2}}
        \pgfpathlineto{\pgfpoint{0pt}{-\pgf@circ@res@other}}
        \pgfpathlineto{\pgfpointanchor{pgf@dac}{B}}
        \pgfpathmoveto{\pgfpointanchor{pgf@dac}{A}}
        \pgfpathlineto{\pgfpoint{\pgf@circ@res@right}{-\pgf@circ@res@other}}
        \pgfusepath{draw}
    }
}

\makeatother


%units

\usepackage[binary-units,retain-explicit-plus]{siunitx}
\DeclareSIUnit \belm {Bm}
\DeclareSIUnit \belfs {BFS}
\DeclareSIUnit \samples {S}
\sisetup{per-mode = symbol}

%plots

\usepackage{pgfplots}
\pgfplotsset{compat=1.12}
\pgfplotsset{every axis plot/.append style={smooth},every axis/.append style={grid=major,legend style={font=\footnotesize}}}
\usepgfplotslibrary{smithchart}
\usepgfplotslibrary{units}
\usepgfplotslibrary{fillbetween}
\pgfplotsset{unit code/.code 2 args={\si{#1#2}}}

\usepgfplotslibrary{external}
\tikzsetexternalprefix{figures/}
\tikzexternalize[mode=list and make]
\tikzexternalwritetomakefile{}
\tikzexternalwritetomakefile{.DELETE_ON_ERROR:}
\tikzexternalwritetomakefile{}

%math

\usepackage[cmex10]{amsmath}
\usepackage{amssymb}
\usepackage{gensymb}

\providecommand{\abs}[1]{\lvert#1\rvert}

%source

\usepackage{listings}
\lstset{language=Python,numbers=left,numberstyle=\tiny,stepnumber=5,numbersep=5pt,frame=single,
breaklines=true,postbreak=\raisebox{0ex}[0ex][0ex]{\ensuremath{\color{red}\hookrightarrow\space}},
captionpos=b,escapeinside={(*}{*)}}

%stuff

\usepackage{url}
\usepackage{tocloft} %table of contents modification
\usepackage{subcaption} %subfigures
\usepackage{placeins} % keep floats in check (=sections where they belong)
\usepackage{booktabs} % nicer tables
\usepackage{lipsum} % TODO remove in final

\usepackage{scrlayer-scrpage}
\pagestyle{scrheadings}
\automark[chapter]{chapter}
\automark*[section]{}

\usepackage[colorlinks,hyperindex,plainpages=false,
pdftitle={FPGA-based Load-Pull Measurement System},
pdfauthor={Gernot Vormayr},
pdfsubject={Diploma thesis},
pdfpagelabels %,hidelinks
]{hyperref}

% lots of acronyms
\usepackage[acronym]{glossaries}

\newacronym{fpga}{FPGA}{field programmable gate array}
\newacronym{iq}{IQ}{in-phase/quadrature-phase}
\newacronym[longplural={general-purpose inputs/outputs}]{gpio}{GPIO}{general-purpose input/output}
\newacronym{fir}{FIR}{finite impulse response}
\newacronym{gui}{GUI}{graphical user interface}
\newacronym{dc}{DC}{direct current}
\newacronym{ui}{UI}{user interface}
\newacronym{rf}{RF}{radio frequency}
\newacronym{pa}{PA}{power amplifier}
\newacronym{mw}{MW}{microwave}
\newacronym{if}{IF}{intermediate frequency}
\newacronym{ram}{RAM}{random access memory}
\newacronym{sparam}{S-parameter}{scattering parameter}
\newacronym{dut}{DUT}{device under test}
\newacronym{emt}{EMT}{electromechanical tuner}
\newacronym{vna}{VNA}{vector network analyzer}
\newacronym{elp}{ELP}{envelope load-pull}
\newacronym{pc}{PC}{personal computer}
\newacronym{adc}{ADC}{analog-to-digital converter}
\newacronym{dac}{DAC}{digital-to-analog converter}

\makeglossaries

\usepackage[noabbrev]{cleveref} %needs to be last

\begin{document}
\begin{titlepage}
    \enlargethispage{1cm}
    \centering
    \vspace*{5cm}
    {\Huge \textbf{Diploma thesis}}\\
    \vspace*{1cm}
    {\Large FPGA-based Load-Pull Measurement System}

    \vspace*{2cm}
    {\large Gernot ~\textsc{Vormayr} \\ 0425210 \\ } ~\\

    \vspace*{2cm}
    {\today } ~\\

    \vfill
    {Supervisors} ~\\\vspace*{0.1cm}
    {Ass.Prof. Dipl.-Ing. Dr.techn. \large Holger ~\textsc{Arthaber}} ~\\
    {Univ.Ass. Dipl.-Ing. Dr.techn. \large Thomas ~\textsc{Faseth}}
    \vspace*{2cm}

    \rule{\linewidth}{0.4pt}
    \begin{minipage}[t]{0.55\linewidth}
        \flushleft
        \begin{large}
            EMCE - Institute of Electrodynamics, Microwave and Circuit Engineering
        \end{large}\\
        Vienna University of Technology
    \end{minipage}
    \hfill
    \begin{minipage}[t]{0.27\linewidth}
        \flushright
        Gusshausstrasse 25\\
        1040 Vienna\\
        www.emce.tuwien.ac.at
    \end{minipage}
    \vspace*{-3pt}
    \rule{\linewidth}{0.4pt}
    \clearpage
\end{titlepage}

\begin{abstract}
    %TODO
    \lipsum[1-5]
\end{abstract}

\begin{otherlanguage}{ngerman}
\begin{abstract}
    %TODO
    \lipsum[1-5]
\end{abstract}
\end{otherlanguage}

\renewcommand{\abstractname}{Acknowledgements}
\begin{abstract}
    %TODO
    \lipsum[1]
\end{abstract}

\tableofcontents

\chapter{Introduction}
\label{chap:introduction}

At \gls{rf} and \gls{mw} frequencies, the circuit theory with lumped elements, where
voltage and current do not vary over the physical dimension of the elements, is of limited
value. The wavelength at these frequencies is of the order of the circuit
element dimensions. This means that transmission line theory has to be used instead \cite{pozar_mw_engineering_2011}.
This theory applies circuit theory to infinitesimal small pieces of the lumped elements
and introduces the concept of forward and backward traveling power waves.

For this reason, instead of impedance- and admittance matrices, \glspl{sparam}
are commonly used in \gls{rf} and \gls{mw} circuit engineering to describe
$N$-ports (see \cref{fig:sparam,eq:sparam} for a 2-port). $a_n$ denotes the incident and
$b_n$ the reflected power wave. Complex valued $S_{nn}$ represents the
part of $a_n$ that gets reflected at port $n$, where as $S_{nm}$ is the part of $a_m$ that
is transmitted to $b_n$. A set of \glspl{sparam} are only valid for a specific frequency, a
characteristic impedance $Z_0$, and a reference plane (port 1 and port 2 in \cref{fig:sparam}).

\begin{figure}[htb]
    \centering
    \begin{tikzpicture}
        \matrix (box)
        [matrix of nodes,%
         nodes in empty cells,
         nodes={dspnodeopen},
         column sep=1cm,
         row sep=2cm]
        {
            |[coordinate]| & &[4cm] & |[coordinate]| \\
            |[coordinate]| & & & |[coordinate]| \\
        };
        \draw[-latex] (box-1-1) node[anchor=east] {$a_1$} -- (box-1-2);
        \draw[-latex] (box-1-2) -- node[above] {$S_{21}$} (box-1-3);
        \draw[-latex] (box-1-3) -- (box-1-4) node[anchor=west] {$b_2$};

        \draw[latex-] (box-2-1) node[anchor=east] {$b_1$} -- (box-2-2);
        \draw[latex-] (box-2-2) -- node[above] {$S_{12}$} (box-2-3);
        \draw[latex-] (box-2-3) -- (box-2-4) node[anchor=west] {$a_2$};

        \draw[-latex] (box-1-2) to[bend left=30] node[right] {$S_{11}$} (box-2-2);
        \draw[latex-] (box-1-3) to[bend right=30] node[left] {$S_{22}$} (box-2-3);

        \draw ($(box-1-2) + (0,0.7cm)$) rectangle ($(box-2-3) - (0,0.7cm)$);

        \draw[dashed] (box-1-2) -- ++(0,1.5cm) node[anchor=south] {Port 1};
        \draw[dashed] (box-1-3) -- ++(0,1.5cm) node[anchor=south] {Port 2};

        \draw[dashed] (box-2-2) -- ++(0,-1cm);
        \draw[dashed] (box-2-3) -- ++(0,-1cm);
    \end{tikzpicture}
    \caption{\Glspl{sparam} of a 2-port}
    \label{fig:sparam}
\end{figure}

\begin{equation}
    \label{eq:sparam}
    \begin{pmatrix} b_1 \\ b_2 \end{pmatrix} = \begin{pmatrix} S_{11} & S_{12} \\ S_{21} & S_{22} \end{pmatrix} \begin{pmatrix} a_1 \\ a_2 \end{pmatrix}
\end{equation}

In a 1-port there is only $S_{11}$. This parameter is equivalent to the reflection
coefficient $\Gamma$ and can be also expressed by the input impedance $Z_{in}$ (see
\cref{eq:s11ref} as shown in \cite{pozar_mw_engineering_2011}). This does not necessarily hold for 2-ports, since for a fully
connected 2-port, also reflections from the device connected at the other port
can be seen.
\begin{equation}
    \label{eq:s11ref} \Gamma = S_{11} = \frac{Z_{in}-Z_0}{Z_{in}+Z_0}
\end{equation}

One way to measure \glspl{sparam} at a specific frequency would be with connecting
a matching impedance $Z_0$. To measure e.g. $S_{11}$ a matching impedance has to
be connected to port 2. According to \cref{eq:s11ref} the reflected wave $a_2$
at port 2 is zero in this case. This means that measurements at port 1 can only see the
reflections caused by $S_{11}$. $S_{11}$ can now be calculated by measuring the incident
and reflected wave at port 1 (see \cref{eq:s11}) \cite{agilent_an_154}.
\begin{align}
    \label{eq:s11} S_{11} & = \left.\frac{b_1}{a_1} \right\rvert_{a_2 = 0}\\
    \label{eq:s21} S_{21} & = \left.\frac{b_2}{a_1} \right\rvert_{a_2 = 0}
\end{align}
Because of connectors and cabling it is often impossible to connect exact matches. Therefor
\glspl{sparam} are measured by sending a power wave into port 1 of the \gls{dut}, measuring
$a_n$ and $b_n$. After that the measurement is repeated with port 2. With measurements from
both ports and calibration measurements (to account for errors caused by connectors and
cabling) it is possible to determine every \gls{sparam}. \Glspl{vna} use this method
and can make automated measurements at various frequencies.

As long as a network behaves linearly with incident small signals, the \glspl{sparam}
fully describe this network at a specific frequency. Thus they can also be used
to model amplifiers, as long as they exhibit controlled and linear behaviour. Since
class-A \glspl{pa} are nearly linear, \glspl{sparam} can be used to describe the small
signal behaviour. But those \glspl{pa} exhibit a very low efficiency of \SI{50}{\percent}.
In modern \gls{rf} and \gls{mw} application the efficiency is increased with input and
output matching networks, that improve the performance. This comes at the cost of non
linear behaviour of the \gls{pa} \cite{ghannouchi_load-pull_2013}.

Because of these non linearities more complex models and verification systems are
needed. One way to measure the characteristics of a non linear system is the
{\it load-pull} (see \cref{fig:load_coeff}). This measurement system presents a
controllable load impedance (output tuner) to the \gls{dut}. The fixed input tuner is
needed to match the input of the \gls{dut} to the source. Load-pull can be used for obtaining the
\gls{dut} characteristics, and for verifying an implementation. This system also
allows to test the \gls{dut} under realistic working conditions.

\begin{figure}[htb]
    \centering
    \begin{tikzpicture}
        \tikzpicturedependsonfile{rfsymbols.tex}
        \draw node[source] (source) {}
              node[tuner,right=of source,label=above:{fixed input tuner}] (ituner) {}
              node[dut,right=of ituner] (dut) {}
              node[tuner,right=2 of dut,label=above:{output tuner}] (otuner) {}
              node[match,right=of otuner,label=above:{$Z_0$}] (match) {}
              node[rground,right=0.3 of match,rotate=90,anchor=center] (ground) {};
        \draw ($(dut)!.5!(otuner)$) node[coordinate] (plane) {};

        \draw (source) -- (ituner) -- (dut) -- (otuner) -- (match) -- (ground);

        \draw [dashed] ($(plane) + (0,2)$) node[anchor=south] {Load Reference Plane} -- ($(plane) - (0,2.5)$);

        \draw [latex-] ($(plane) + (0,1.5)$) -- ++(-0.5,0) node[anchor=east] {$Z_L$};

        \draw [-latex] ($(plane) + (0.25,-1)$) node[anchor=west] {$a_2$}-- ++(-0.5,0);
        \draw [latex-] ($(plane) + (0.25,-1.5)$) node[anchor=west] {$b_2$}-- ++(-0.5,0);
        \draw [latex-] ($(plane) + (0,-2)$) -- ++(-0.5,0) node[anchor=east] {$\Gamma_L$};

    \end{tikzpicture}
    \caption{Load reflection coefficient}
    \label{fig:load_coeff}
\end{figure}

The \cref{eq:gl,eq:zl} show the relations between the load reflection coefficient $\Gamma_L$,
the incident wave $a_2$, the reflected
wave $b_2$, and the load impedance $Z_L$ at port 2 of \cref{fig:sparam,fig:load_coeff}. $Z_0$
is the characteristic impedance of the system \cite{hashmi_highly_2011}.
\begin{align}
    \label{eq:gl} \Gamma_L & = \frac{a_2}{b_2} \\
    \label{eq:zl} \Gamma_L & = \frac{Z_L-Z_0}{Z_L+Z_0}
\end{align}
The output tuner in \cref{fig:load_coeff} synthesizes an appropriate $\Gamma_L$
either by varying the phase and magnitude of the reflected wave $a_2$ or by varying
the load impedance $Z_L$. This means, that it is possible to build a load-pull
setup by either using a passive tuner, or feeding a modified wave back to the
\gls{dut}.

There are various types of load-pull measurement setups, which have different
characteristics. One important features is the repeatability of reflection
coefficients. The repeatability is needed to ensure accurate application
specific device models. Another important factor is the tuning range and
distribution (e.g. \cref{fig:range_passive}). Usually passive load-pull
systems have a more limited tuning range than active ones, but provide a
better repeatability \cite{ghannouchi_load-pull_2013}. Tuner speed and tuner
resolution is an additional trade off. High resolution is needed since \glspl{pa} are
highly sensitive to impedance variations. However a high resolution incurs
a slow tuner speed. An exemplary resolution can be seen in \cref{fig:generic_emt}.
Power handling capability is another extremely important factor. The load-pull
setup has to be capable to sustain the power presented to the tuner without
damage. Which load-pull setup to choose for a specific \gls{dut} depends on
these requirements.

\begin{figure}[htb]
    \begin{subfigure}[b]{.5\linewidth}
        \centering
        \begin{tikzpicture}
            \begin{smithchart}[width=.9\linewidth]
                \addplot[blue,is smithchart cs,mark=o,only marks] file {testdata/generic_emt.data};
            \end{smithchart}
        \end{tikzpicture}
        \caption{\Gls{emt}}
        \label{fig:generic_emt}
    \end{subfigure}%
    \begin{subfigure}[b]{.5\linewidth}
        \centering
        \begin{tikzpicture}
            \begin{smithchart}[width=.9\linewidth]
                \node [circle,draw,pattern=crosshatch,minimum size=.6\linewidth] (0,0) {};
            \end{smithchart}
        \end{tikzpicture}
        \caption{Passive load-pull system}
        \label{fig:range_passive}
    \end{subfigure}
    \caption{Generic representation of achievable tuning range/points passive load-pull}
\end{figure}

Passive load-pull systems are based on the block diagram in \cref{fig:load_coeff}. Depending on
the desired measurements, additional circuit elements like directional couplers,
power meters, and oscilloscopes are needed. For higher power measurements additional
amplifiers and attenuators, to reduce the output power to acceptable levels, might also
be needed. These additional components, the cabling, and connectors add additional loss
on the reflection path, leading to achievable reflection levels $\abs{\Gamma_L} < 1$ with
a maximum usually around 0.75 \cite{de_groote_introduction_2008} (see \cref{fig:range_passive}).

Typical tuners used, consist of a transmission line and a probe, that introduces a mismatch
by adding a parallel susceptance. Varying the position of the probe along the transmission
line changes the phase of the impedance mismatch and the distance the magnitude \cite{hashmi_highly_2011}.
Automated positioning can be achieved by adding motors. These tuners are called \glspl{emt}.
Those have to be calibrated before use and achieve reflection levels like in
\cref{fig:generic_emt} with up to 10000 points \cite{ghannouchi_load-pull_2013}.

If higher $\abs{\Gamma_L}$-levels  are needed (even levels $> 1$), active load-pull setups
can be used. There are two categories: closed- and open loop.

\begin{figure}[htb]
    \centering
    \begin{tikzpicture}
        \tikzpicturedependsonfile{rfsymbols.tex}
        \draw node[dut] (dut) {}
              node[circulator,right=2 of dut] (circ) {}
              node[vmatch,right=of circ,label=above:{Attenuator},yshift=1cm] (att) {}
              node[vphase,right=1.5 of att,label=above:{Phase shift}] (phase) {}
              node[amplifier,right=of circ,label=above:{Amplifier},yshift=-1cm] (amp) {};

        \draw ($(dut)!.5!(circ)$) node[coordinate] (plane) {};

        \draw (dut) -- (circ.A);
        \draw (circ.B) -- ($(circ.B |- att)!.5!(att)$) node[coordinate] (upper) {} -- (att) -- (phase) -- ++(1,0) |- (amp) -- ($(circ.C |- amp)!.5!(amp)$) node[coordinate] (lower) {} -- (circ.C);

        \draw [-latex] ($(circ.B) + (-0.2,0.2)$) -- node[sloped,above] {$b_2$} ($(upper) + (-0.2,0.2)$);
        \draw [latex-] ($(circ.C) + (-0.2,-0.2)$) -- node[sloped,below] {$a_2$} ($(lower) + (-0.2,-0.2)$);

        \draw [dashed] ($(plane) + (0,2)$) node[anchor=south] {Load Reference Plane} -- ($(plane) - (0,2.5)$);

        \draw [latex-] ($(plane) + (0,1.5)$) -- ++(-0.5,0) node[anchor=east] {$Z_L$};

        \draw [-latex] ($(plane) + (0.25,-1)$) node[anchor=west] {$a_2$}-- ++(-0.5,0);
        \draw [latex-] ($(plane) + (0.25,-1.5)$) node[anchor=west] {$b_2$}-- ++(-0.5,0);
        \draw [latex-] ($(plane) + (0,-2)$) -- ++(-0.5,0) node[anchor=east] {$\Gamma_L$};
    \end{tikzpicture}
    \caption{Active closed loop load-pull block diagram}
    \label{fig:active_closed_loop}
\end{figure}

Active closed loop load-pull setups generate a modified reflected wave $a_2$ by
modifying the phase and magnitude of $b_2$ and feeding it back to the \gls{dut}.
The functional block diagram which can be seen in \Cref{fig:active_closed_loop}
is an example of such an active closed loop load-pull setup. With the circulator
the power wave $b_2$ is fed to a variable attenuator, a phase shifter and a
amplifier. These elements enable modifying the phase and the magnitude of $b_2$,
which is again fed back to the \gls{dut} with the circulator. Because of limited
isolation provided by real world circulators this system will oscillate if the loop gain
exceeds the isolation \cite{ghannouchi_load-pull_2013}. This can be circumvented with
an isolator after the amplifier. Another way to improve the stability of the system
is to introduce a bandpass filter into the loop.

\begin{figure}[htb]
    \centering
    \begin{tikzpicture}
        \tikzpicturedependsonfile{rfsymbols.tex}
        \draw node[dut] (dut) {}
              node[isolator,right=2 of dut,rotate=180,anchor=east] (iso) {}
              node[amplifier,right=of iso.west,label=above:Amplifier] (amp) {}
              node[vphase,right=of amp,label=below:{Phase Shift}] (shift) {}
              node[vmatch,right=of shift,label=above:{Attenuator}] (att) {}
              node[source,right=of att,label=below:{Phase Locked Source}] (source) {};

        \draw ($(dut)!.5!(iso)$) node[coordinate] (plane) {};

        \draw (dut) -- (iso) -- (amp) -- (shift) -- (att) -- (source);
        \draw [latex-] ($(iso.west) + (0.2,-0.2)$) -- node[below] {$a_2$} ($(amp.west) + (-0.2,-0.2)$);

        \draw [dashed] ($(plane) + (0,2)$) node[anchor=south] {Load Reference Plane} -- ($(plane) - (0,2.5)$);

        \draw [latex-] ($(plane) + (0,1.5)$) -- ++(-0.5,0) node[anchor=east] {$Z_L$};

        \draw [-latex] ($(plane) + (0.25,-1)$) node[anchor=west] {$a_2$}-- ++(-0.5,0);
        \draw [latex-] ($(plane) + (0.25,-1.5)$) node[anchor=west] {$b_2$}-- ++(-0.5,0);
        \draw [latex-] ($(plane) + (0,-2)$) -- ++(-0.5,0) node[anchor=east] {$\Gamma_L$};
    \end{tikzpicture}
    \caption{Active open loop load-pull block diagram}
    \label{fig:active_open_loop}
\end{figure}

Active open loop load-pull setups work by generating a phase coherent wave with
an external signal generator (see \cref{fig:active_open_loop}). The open loop
approach has the advantage, that it can't oscillate, since there is no closed loop.
It works by generating a signal with a source, that is locked to the generator driving the
source port of the \gls{dut}. Phase and magnitude can be adjusted with the attenuator and
phase shifter. A disadvantage of the open loop system is that the generated wave at
the load generator is equivalent to the one on the source side. This implies that the reflection coefficient
$\Gamma_L$ depends on the large-signal \glspl{sparam} $S_{21}$ and $S_{22}$ of the \gls{dut}.
Therefore, an iterative approach is needed to be able to achieve specific $\Gamma_L$ \cite{muller_comparison_1994}.

The following chapters introduce an active closed loop load-pull system, which
is built with a digital attenuator, phase shifter, and loop filter implemented
in a \gls{fpga}.

% ===========================================================================

\chapter{\glsentryshort{fpga}-based Load-Pull Measurement System}

The active closed loop load-pull measurement system mentioned in
\cref{chap:introduction} can't synthesize harmonically independent
reflection coefficients, which are needed in the design of high-efficiency
\glspl{pa}. Furthermore it depicts a strong interdependency of magnitude
and phase of the reflection coefficient $\Gamma_L$ on both phase shifter
and attenuator settings \cite{pozar_mw_engineering_2011}. To overcome these
limitations an \gls{elp} was designed.

\begin{figure}[htb]
    \centering
    \begin{tikzpicture}
        \tikzpicturedependsonfile{rfsymbols.tex}
        \tikzstyle{every node}=[font=\footnotesize]
        \draw node[dut] (dut) {}
              node[circulator,right=2 of dut,label=below:circ] (circ) {}

              node[coordinate,right=of circ,yshift=1.5cm] (uppernode) {}
              node[coordinate,right=of circ,yshift=-1.5cm] (lowernode) {}

              node[amplifier,right=0.5 of lowernode,label=above:amp] (amp) {}
              node[mixer,right=of amp,label=below:mix2] (upmix) {}

              (uppernode -| upmix) node[mixer,label=above:mix1] (downmix) {}

              ($(upmix)!.5!(downmix)$) node[mixer] (mul) {}
              node[rotate=90,anchor=south] at (mul.west) {mul}

              node[right=0.5 of mul] (gamma) {$\Gamma_{set} = X + j Y$}
              node[source,right=0.5 of gamma,scale=0.7,label=right:$f_0$] (lo) {};

        \draw ($(dut)!.5!(circ)$) node[coordinate] (plane) {};

        \draw [latex-] ($(amp.west |- uppernode) + (-0.2,0.2)$) -- node[sloped,above] {$b_2$} ($(uppernode) + (-0.4,0.2)$);
        \draw [-latex] ($(amp.west) + (-0.2,-0.2)$) -- node[below] {$a_2$} ($(lowernode) + (-0.4,-0.2)$);

        { [rounded corners=2pt]
            \draw (dut) -- (circ.A);
            \draw [-latex] (circ.B) -- ($(circ.B |- uppernode)!.5!(uppernode)$) node[coordinate] (upper) {} -- (downmix);
            \draw [-latex] (amp) -- ($(circ.C |- amp)!.5!(lowernode)$) node[coordinate] (lower) {} -- (circ.C);
            \draw [-latex] (lo) |- (downmix);
            \draw [-latex] (lo) |- (upmix);
        }
        \draw [-latex] (gamma) -- (mul);
        { [start chain,every on chain/.style={join=by -latex}]
            \chainin (downmix);
            { [every on chain/.style={join=by {double,-latex}}]
                \chainin (mul);
                \chainin (upmix);
            }
            \chainin (amp);
        }

        \draw [dashed] ($(plane) + (0,2)$) node[anchor=south] {Load Reference Plane} -- ($(plane) - (0,3)$);

        \draw [latex-] ($(plane) + (0,1.5)$) -- ++(-0.5,0) node[anchor=east] {$Z_L$};

        \draw [-latex] ($(plane) + (0.25,-1.5)$) node[anchor=west] {$a_{2}$}-- ++(-0.5,0);
        \draw [latex-] ($(plane) + (0.25,-2)$) node[anchor=west] {$b_{2}$}-- ++(-0.5,0);
        \draw [latex-] ($(plane) + (0,-2.5)$) -- ++(-0.5,0) node[anchor=east] {$\Gamma_L$};
    \end{tikzpicture}
    \caption{Generic block diagram \gls{elp}}
    \label{fig:elp}
\end{figure}

Instead of traditional active closed loop load-pull, this system generates the
load coefficient at baseband or \gls{if}. The basic principle of this system
can be seen in \cref{fig:elp}. A circulator is used to split the incident and
reflected wave. The mixer \textit{mix1} is used for shifting the spectrum to
the baseband and for \gls{iq} demodulation. This \gls{iq} signal is then
multiplied by the complex valued $\Gamma_{set}$ (\textit{mul}), thus creating a specific
reflection coefficient $\Gamma_L$. Mixer \textit{mix2} is used for modulating
the \gls{iq} signal and shifting the signal back to the desired \gls{rf} \cite{williams_experimental_2005}.

The multiplication can be done in the analog domain, as has been described
by \cite{williams_experimental_2005}. This has the disadvantage, that
an additional image will be generated by the multiplicator \textit{mul},
if there are amplitude imbalances in the output
of the demodulator \textit{mix1}. Since this image is very close to the
carrier signal, it can't be filtered away \cite{hashmi_agile_2010}.

\begin{figure}[htb]
    \centering
    \resizebox{\linewidth}{!}{
    \begin{tikzpicture}
        \pgfdeclarelayer{foreground}
        \pgfsetlayers{background,main,foreground}
        \tikzpicturedependsonfile{rfsymbols.tex}
        \tikzstyle{every node}=[font=\footnotesize]
        \draw node[dut] (dut) {}
              node[dircoupler,right=2 of dut,label=below:dir] (dirvna) {}
              node[oscilloscope,above=of dirvna.A2,anchor=A1] (oszivna) {}
              node[circulator,right=of dirvna,label=below:circ] (circ) {}

              node[coordinate,right=of circ,yshift=1.5cm] (uppernode) {}
              node[coordinate,right=of circ,yshift=-1.5cm] (lowernode) {}

              (lowernode) node[amplifier,label=above:amp] (amp) {}
              node[mixer,right=of amp,label=above:mix3] (upmix) {}
              node[lowpass,right=of upmix,label=above:f3] (antialias) {}
              node[adc,right=of antialias,label=above:dac] (dac) {}
              node[empty,right=of dac,label=above:buffer2] (outbuf) {}
              node[mixer,right=of outbuf] (mul) {}
              node[rotate=90,anchor=north] at (mul.east) {mul}

              (uppernode -| upmix) node[mixer,label=below:mix1] (downmix) {}
              node[bandpass,right=of downmix,label=below:f1] (bp) {}
              node[adc,right=of bp,label=below:adc] (adc) {}
              node[empty,right=of adc,label=below:buffer1] (inbuf) {}
              node[mixer,right=of inbuf] (shift) {}
              node[rotate=90,anchor=north] at (shift.east) {mix2}

              ($(shift)!.5!(mul)$) node[allpass,rotate=90] (H) {}
              node[rotate=90,anchor=north] at (H.south) {f2}

              node[source,above=of downmix.center,scale=0.7,label=right:$f_0 - \SI{70}{\mega\hertz}$] (downsource) {}
              node[source,above=of adc.center,scale=0.7,label={[fill=white]right:$\SI{100}{\mega\hertz}$}] (sample) {}
              node[source,above=of shift.center,scale=0.7,label=right:$\SI{30}{\mega\hertz}$] (shiftsource) {}
              node[source,below=of upmix.center,scale=0.7,label=right:$f_0$] (upsource) {}
              node[below=of mul.center] (gamma) {$\Gamma_{set} = X + jY$};

        \draw ($(dut)!.5!(dirvna)$) node[coordinate] (plane) {};

        { [rounded corners=2pt]
            \draw (dut) -- (dirvna) -- (circ.A);
            \draw [-latex] (circ.B) -- ($(circ.B |- uppernode)!.5!(uppernode)$) node[coordinate] (upper) {} -- (downmix);
            \draw (dirvna.A2) -- (oszivna.A1);
            \draw (dirvna.B2) |- ($(dirvna.B2)!.7!(oszivna.A2)$) node[coordinate] (oszimiddle) {} -| (oszivna.A2);
            \draw [-latex] (amp) -- ($(circ.C |- amp)!.5!(lowernode)$) node[coordinate] (lower) {} -- (circ.C);
        }
        { [-latex]
            \draw (downsource) -- (downmix);
            \draw (sample) -- (adc);
            \draw (shiftsource) -- (shift);
            \draw (gamma) -- (mul);
            \draw (upsource) -- (upmix);
        }
        { [latex-,dashed,every node/.style={font=\footnotesize}]
            \foreach \device in {oszivna,downsource,sample,shiftsource} {
                \draw (\device) -- ++(0,1) node[anchor=south] {ref};
            }
            \draw (upsource) -- ++(0,-1) node[anchor=north] {ref};
        }
        { [start chain,every on chain/.style={join=by -latex}]
            \chainin (downmix);
            \chainin (bp);
            \chainin (adc);
            \chainin (inbuf);
            \chainin (shift);
            { [every on chain/.style={join=by {double,-latex}}]
                \chainin (H);
                \chainin (mul);
                \chainin (outbuf);
                \chainin (dac);
                \chainin (antialias);
                \chainin (upmix);
            }
            \chainin (amp);
        }
        { [on background layer,every path/.style={dotted,decorate,decoration=random steps,segment length=2mm}]
            \draw ($(dirvna.B1)!.5!(circ.A) + (0,4)$) -- ++(0,-8.5) node[coordinate] (leftsplit) {};
            \draw ($(adc.east)!.5!(outbuf.west) + (0,4)$) -- ++(0,-8.5) node[coordinate] (rightsplit) {};
        }

        \draw (leftsplit) node[anchor=base east] {One Port \gls{vna}}
              (leftsplit) node[anchor=base west] {Analog}
              (rightsplit) node[anchor=base west] {Digital}
              (rightsplit) node[anchor=base east] {Analog};

        \draw [-latex] ($(dirvna.A2) + (-0.2,0.1)$) -- node[base left] {$b_{2,e}$} ($(oszivna.A1 |- oszimiddle) - (0.2,0.1)$);
        \begin{pgfonlayer}{foreground}
            \draw [-latex] ($(dirvna.B2) + (0.2,0.1)$) -- ($(oszimiddle -| dirvna.B2) + (0.2,-0.1)$);
        \end{pgfonlayer}
        \draw ($(oszimiddle -| dirvna.B2) + (0.2,-0.1)$) node[base right,fill=white] {$a_{2,e}$};

        \draw [dashed] ($(plane) + (0,4)$) node[anchor=south] {Load Reference Plane} -- ($(plane) - (0,4.5)$);

        \draw [latex-] ($(plane) + (0,1.5)$) -- ++(-0.5,0) node[anchor=east] {$Z_L$};

        \draw [-latex] ($(plane) + (0.25,-1.5)$) node[anchor=west] {$a_{2,dut}$}-- ++(-0.5,0);
        \draw [latex-] ($(plane) + (0.25,-2)$) node[anchor=west] {$b_{2,dut}$}-- ++(-0.5,0);
        \draw [latex-] ($(plane) + (0,-2.5)$) -- ++(-0.5,0) node[anchor=east] {$\Gamma_L$};
    \end{tikzpicture}
    }
    \caption{System overview}
    \label{fig:overall_hf}
\end{figure}

To overcome these limitations, a digital \gls{elp} similar to the one in
\cite{hashim_active_2008} was developed. The design in \cite{hashim_active_2008}
uses direct conversion, which creates additional \gls{dc} components in the \gls{iq}
signals. Since these have to be removed by filters, the band around \SI{0}{\hertz} is
not usable for reflection generation. Additionally it used a variable delay line in
the \gls{fpga}, to compensates the phase difference of multiple carriers. The design in
\cref{fig:overall_hf} was developed with these limitations in mind. Instead of direct
conversion a superheterodyne design is used. Furthermore the delay line is replaced by
a fully configurable filter \textit{f2}. It consists of three distinct parts:

\begin{enumerate}
    \item A one port \gls{vna}, to be able to measure the current $\Gamma_L$. This part is
needed, to be able to reach a specific $\Gamma_{L,target}$ iteratively. It is
described in detail in \cref{sec:vna}, whereas the iterative algorithm can be found
in \cref{sec:matlab}.

    \item The analog part, which contains the mixers for the frequency shifting, necessary
        filters, \gls{dac}, \gls{adc}, and an amplifier. A detailed description can be found
        in \cref{sec:analog}.

    \item A digital processing chain implemented in a \gls{fpga}, which is controlled with
a \gls{pc} running Matlab. The \gls{fpga} implementation is described in \cref{chap:fpga},
the software running on the processor contained in the \gls{fpga}, as well as the Matlab
code, in \cref{chap:software}.
\end{enumerate}

% ---------------------------------------------------------------------------

\section{One Port \glsentryshort{vna}}
\label{sec:vna}

\begin{itemize}
    \item describe network analyzer part
    \item describe error box
    \item describe calibration
\end{itemize}

\begin{figure}[htb]
    \centering
    \begin{tikzpicture}
        \matrix (box)
        [matrix of nodes,%
         nodes in empty cells,
         nodes={dspnodeopen},
         column sep=1cm,
         row sep=2cm]
        {
            |[coordinate]| & &[1.5cm] &[4cm] & |[coordinate]| \\
            |[coordinate]| & & & & |[coordinate]| \\
        };
        \draw[-latex] (box-1-2) -- node[above] {$b_{2,dut}=a_{1,e}$}  (box-1-3);
        \draw[-latex] (box-1-3) -- node[above] {$1$} (box-1-4);
        \draw[-latex] (box-1-4) -- (box-1-5) node[anchor=west] {$b_{2,e}$};

        \draw[latex-] (box-2-2) -- node[above] {$a_{2,dut}=b_{1,e}$} (box-2-3);
        \draw[latex-] (box-2-3) -- node[above] {$S_{12}$} (box-2-4);
        \draw[latex-] (box-2-4) -- (box-2-5) node[anchor=west] {$a_{2,e}$};

        \draw[-latex] (box-1-3) to[bend left=30] node[right] {$S_{11}$} (box-2-3);
        \draw[latex-] (box-1-4) to[bend right=30] node[left] {$S_{22}$} (box-2-4);

        \draw[-latex] (box-1-2) to[bend right=30] node[left] {$\Gamma_L$} (box-2-2);

        \draw ($(box-1-3) + (0,0.7cm)$) rectangle ($(box-2-4) - (0,0.7cm)$);
        \draw ($(box-1-1) + (0,0.7cm)$) rectangle ($(box-2-2) - (0,0.7cm)$);

        \draw[dashed] (box-1-3) -- ++(0,1.5cm) node[anchor=south] {Load Reference Plane};
        \draw[dashed] (box-1-4) -- ++(0,1.5cm) node[anchor=south] {Oscilloscope};

        \draw[dashed] (box-2-3) -- ++(0,-1cm);
        \draw[dashed] (box-2-4) -- ++(0,-1cm);
    \end{tikzpicture}
    \caption{Errorbox for the one port vector network analyzer}
    \label{fig:errorbox}
\end{figure}

%TODO fix equations according to errorbox

\begin{align}
    b_2 & = a_1 + S_{22} a_2\\
    b_1 & = S_{11} a_1 + S_{12} a_2 \\
    a_2 & = \Gamma_L b_2
\end{align}

\begin{equation}
    \Gamma_L = \frac{S_{11} - \frac{b_1}{a_1}}{S_{11} S_{22} - S_{22}\frac{b_1}{a_1} - S_{12}}
\end{equation}

\begin{align}
\Gamma_1 & = \frac{a_1}{b_1} \\
\begin{split}
g & = (\Gamma_{1_O} - \Gamma_{1_S}) \Gamma_{1_M} \Gamma_{L_O} \Gamma_{L_S} + \Gamma_{L_M} ((\Gamma_{1_M} - \Gamma_{1_O}) \Gamma_{1_S} \Gamma_{L_O} + (\Gamma_{1_S} - \Gamma_{1_M}) \Gamma_{1_O} \Gamma_{L_S})
\end{split}\\
S_{11} & = \frac{(\Gamma_{1_O} - \Gamma_{1_S}) \Gamma_{L_O} \Gamma_{L_S} + \Gamma_{L_M} ((\Gamma_{1_M} - \Gamma_{1_O}) \Gamma_{L_O} + (\Gamma _{1_S} - \Gamma _{1_M}) \Gamma_{L_S})}{g} \\
S_{12} & = \frac{(\Gamma_{1_O} - \Gamma_{1_M}) (\Gamma_{1_M} - \Gamma_{1_S}) (\Gamma_{1_O} - \Gamma_{1_S}) (\Gamma_{L_M} - \Gamma_{L_O}) (\Gamma_{L_M} - \Gamma_{L_S}) (\Gamma_{L_O} - \Gamma_{L_S})}{g^2} \\
S_{22} & = \frac{(\Gamma_{1_S} - \Gamma_{1_O}) \Gamma_{1_M} \Gamma_{L_M} + (\Gamma_{1_M} - \Gamma_{1_S}) \Gamma_{1_O} \Gamma_{L_O} + (\Gamma_{1_O} - \Gamma_{1_M}) \Gamma_{1_S} \Gamma_{L_S}}{g}
\end{align}

% ---------------------------------------------------------------------------

\section{Analog Part}
\label{sec:analog}

\begin{itemize}
    \item describe spectra
    \item why 70
    \item why lower part
\end{itemize}

% ===========================================================================

\chapter{\glsentryshort{fpga} Implementation}
\label{chap:fpga}

\begin{itemize}
    \item describe paper \cite{hashim_active_2008}
    \item improvements done
\end{itemize}

\section{Data Acquisition}
\section{Overlap Add}
\section{SMBV Interface}
\section{Processor Interface}
\chapter{Software Implementation}
\label{chap:software}
\section{Kernel Module}
\section{Network Server and Web-Interface}
\section{Matlab Driver}
\label{sec:matlab}

\chapter{Verification of the Measurement System}
\begin{tikzpicture}[node distance=0.6]
    \tikzpicturedependsonfile{rfsymbols.tex}
    \tikzstyle{every node}=[font=\footnotesize]
    \draw node[dut] (dut) {}
          node[dircoupler,right=1 of dut] (dirvna) {}
          node[dircouplera,right=of dirvna] (dircirc) {}
          node[attenuator,right=of dircirc,label=below:\scriptsize\SI{10}{\deci\bel}] (attcirc) {}
          node[vectorgenerator,right=of attcirc,anchor=out,label=below:\scriptsize\SI{900}{\mega\hertz}] (gen) {}

          node[lowpass,above=of attcirc,label=below:\scriptsize\SI{81}{\mega\hertz}] (alias) {}
          node[mixer,rotate=180,left=of alias,anchor=in,scale=0.5] (mixer) {}
          node[adc,right=of alias] (adc) {}
          node[generator,above=of mixer.in 2] (lo) {}
          node[rotate=90,anchor=north] at (lo.east) {\scriptsize\SI{830}{\mega\hertz}}

          node[generator,above=of adc] (sample) {}
          node[rotate=90,anchor=north] at (sample.east) {\scriptsize\SI{100}{\mega\hertz}}

          node[oscilloscope,above=of dirvna.A2,anchor=A1] (oszivna) {}

          node[mixer,right=1 of gen.in,scale=0.4,rotate=180,anchor=out] (mult) {}
          (mult |- adc) node[mixer,scale=0.4] (iqdemod) {}
          ($(mult)!.5!(iqdemod)$) node[allpass,rotate=-90,scale=0.7] (filter) {}
          node[rotate=90,anchor=north] at (filter.north) {H}

          node[vsourcesinshape,scale=.5,rotate=90,right=of iqdemod.out,anchor=center] (diglo) {}
          node[rotate=90,anchor=north] at (diglo.south) {\scriptsize\SI{30}{\mega\hertz}}
          (mult -| diglo) node (gamma) {$\Gamma$};

    \draw [rounded corners=2pt]
          (dut.B) -- (dirvna.A1)
          (dirvna.B1) -- (dircirc.A1)
          (dircirc.B1) -- (attcirc.A)
          (attcirc.B) -- (gen.out)

          (dirvna.A2) -- (oszivna.A1)
          (dirvna.B2) |- ($(dirvna.B2)!.5!(oszivna.A2)$) -| (oszivna.A2)

          (dircirc.A2) |- (mixer.out)
          (mixer.in) -- (alias.A)
          (alias.B) -- (adc.A)

          (mixer.in 2) -- (lo)

          (sample) -- (adc) -- (iqdemod.in)
          (iqdemod.in 2) -- (filter) -- (mult.in 2)
          (mult.out) -- (gen.in)

          (diglo) -- (iqdemod.out)
          (gamma) -- (mult.in);

    \begin{scope}[rounded corners=2pt]
        \draw [<-]
        (sample.north) |- ($(sample.north) + (5pt,0.4)$) node[coordinate] (merge) {} -| ($(gen.north west)!.85!(gen.north east)$) node[coordinate] (refgen) {};
        \draw [->] (merge) -| (lo.north);
        \draw [->] (merge) -| (oszivna.north);
    \end{scope}

    \draw ($(refgen |- oszivna.north)!.5!(oszivna.north)$) node[coordinate] (refa) {}
          (refa |- merge) node[anchor=south] {\scriptsize\SI{10}{\mega\hertz} Reference clock};

    \node[draw,fit=(iqdemod) (filter) (mult) (diglo),rounded corners=4pt,inner xsep=12pt,inner ysep=8pt,label=above:FPGA] {};

\end{tikzpicture}
\section{One Port \glsentryshort{vna}}
\section{Reflection Measurements}
\begin{tikzpicture}
    \begin{smithchart}[width=7cm,clip=false]
        \foreach \i in {1,...,11}{
            \foreach \j in {1,...,11}{
                \addplot[blue,is smithchart cs] file {testdata/filter/-1+0i/\i,\j.data};
            }
        }
        \addplot[red,is smithchart cs,mark=*,only marks] coordinates {(-1,0)};
    \end{smithchart}
\end{tikzpicture}\\
\begin{tikzpicture}
    \begin{smithchart}[width=7cm,clip=false]
        \foreach \i in {1,...,11}{
            \foreach \j in {1,...,11}{
                \addplot[blue,is smithchart cs] file {testdata/filter/+1+0i/\i,\j.data};
            }
        }
        \addplot[red,is smithchart cs,mark=*,only marks] coordinates {(+1,0)};
    \end{smithchart}
\end{tikzpicture}\\
\begin{tikzpicture}
    \begin{smithchart}[width=7cm,clip=false]
        \foreach \i in {1,...,11}{
            \foreach \j in {1,...,11}{
                \addplot[blue,is smithchart cs] file {testdata/filter/mean1/\i,\j.data};
            }
        }
        \addplot[red,is smithchart cs,mark=*,only marks] coordinates {(-1,0) (+1,0)};
    \end{smithchart}
\end{tikzpicture}

\section{Phase Drift}
\begin{tikzpicture}
    \begin{axis}[
            ylabel={angle},
            xlabel={seconds},
            x unit={\s},
            width=\linewidth
        ]
        \addplot[blue] table {testdata/phase/single.data};
    \end{axis}
\end{tikzpicture}

\begin{tikzpicture}
    \pgfplotstableread{testdata/phase/all.data}\phaseall
    \begin{axis}[
            ylabel={angle},
            xlabel={seconds},
            x unit={\s},
            width=\linewidth
        ]
        \addplot[blue] table {\phaseall};
        \addlegendentry{\gls{vna}}
        \addplot[green] table[x index=0,y index=2] {\phaseall};
        \addlegendentry{R\&S SMIQ}
        \addplot[red] table[y index=3] {\phaseall};
        \addlegendentry{R\&S SMGU}
        \addplot[black] table[y index=4] {\phaseall};
        \addlegendentry{R\&S SMBV}
    \end{axis}
\end{tikzpicture}
\chapter{Conclusions and Outlook}

\begin{appendix}
    \chapter{Hardware Reference}
    \section{Physical Overview}
    \section{Memory Map \& Register Assignment}
    \chapter{Software Reference}
    \section{Kernel Module Usage}
    \section{Protocols}
    \section{Matlab Classes}
    \section{Usage Examples}
    \chapter{Build Instructions}
    \section{Hardware}
    \section{Software}

    \printglossary[type=\acronymtype]

    \listoffigures

    \bibliographystyle{IEEEtran}
    \bibliography{main}

\begin{otherlanguage}{ngerman}
    \chapter*{Code of Conduct}
    Hiermit erkl\"are ich, dass die vorliegende Arbeit ohne unzul\"assige Hilfe Dritter und ohne Benutzung
    anderer als der angegebenen Hilfsmittel angefertigt wurde. Die aus anderen Quellen oder indirekt
    \"ubernommenen Daten und Konzepte sind unter Angabe der Quelle gekennzeichnet.
    Die Arbeit wurde bisher weder im In- noch im Ausland in gleicher oder in \"ahnlicher Form in anderen
    Pr\"ufungsverfahren vorgelegt.

    \par\noindent\makebox[7cm]{\hrulefill}      \hfill\makebox[5cm]{\hrulefill}%
    \par\noindent\makebox[7cm][l]{Unterschrift} \hfill\makebox[5cm][l]{Datum}%
\end{otherlanguage}

    \end{appendix}
\end{document}

