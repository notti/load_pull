\documentclass[11pt,technote,a4paper,onecolumn,dvips]{IEEEtran}

\usepackage{etex}
\usepackage{register}
\usepackage[utf8]{inputenc}
\usepackage[T1]{fontenc}
\usepackage{pstricks}
\usepackage{pstricks-add}
\usepackage{pst-circ}
\usepackage{pst-plot}
\usepackage{pst-sigsys}
\usepackage{calc}
\usepackage{tikz-timing}
\usepackage{siunitx}
\usepackage{cite}
\usepackage{url}
\usepackage{dblfloatfix}
\usepackage{listings}
\usepackage{afterpage}

\usepackage[colorlinks,hyperindex,plainpages=false,
pdftitle={Specialized DSP in FPGA with Processor Bindings},
pdfauthor={Gernot Vormayr},
pdfsubject={Bac thesis},
pdfkeywords={},
pdfpagelabels,
pagebackref,
bookmarksopen=false
]{hyperref}

\usepackage[nameinlink]{cleveref}

\markboth{EMCE-Vienna University of Technology,~Sep 2014}{Gernot Vormayr: DSP in FPGA}

\newcommand{\legendline}[2]{\begingroup\setbox0=\hbox{\textcolor{#1}{\rule{4pt}{#2}}}\parbox{\wd0}{\box0}\endgroup}

\newcommand{\signal}[1]{{\ttfamily #1}}
\newcommand{\module}[1]{{\ttfamily\bfseries #1}}
\newcommand{\clk}[1]{{\itshape\ttfamily #1}}

\crefname{Regfloat}{register}{registers}
\Crefname{Regfloat}{Register}{Registers}

\lstdefinelanguage{ucf}{%
 morekeywords={INST,AREA_GROUP,RANGE,LOC},
 sensitive=false,
 morecomment=[l][\#],
 morestring=[b]",
}

\lstset{numbers=left,numberstyle=\tiny,stepnumber=2,numbersep=5pt,frame=single,
breaklines=true,postbreak=\raisebox{0ex}[0ex][0ex]{\ensuremath{\color{red}\hookrightarrow\space}},
captionpos=b,basicstyle=\ttfamily}

\newcommand*{\lstitem}[1]{
  \setbox0\hbox{\lstinline{#1}}  
  \item[\usebox0]  
  \hfill \\
}

\begin{document}


\title{\huge Specialized DSP in FPGA with Processor Bindings}
\author{\IEEEauthorblockN{Gernot Vormayr - 0425210}\\
\IEEEauthorblockA{Vienna University of Technology\\
\url{gvormayr@gmail.com}}}

\maketitle

\begin{abstract}
This bachelor thesis presents a step by step guide for implementing a
specialized DSP functionality on an ML507 Board. The whole guide
covers implementing the actual hardware, the interface to the integrated
PowerPC processor, building the linux kernel and a small base system,
writing a kernel driver for the interface, and implementing a small
web interface in python.
\end{abstract}

\renewcommand{\contentsname}{\small{Table of Contents}}
\tableofcontents

\section{Introduction}
\begin{figure}[bt]
    \centering
    \begin{pspicture}(0,-1.5)(13.25,2)
        \pssignal(0,0){x}{ADC}
        \psblock(2,0){avg}{AVG}
        \pssignal(3.5,-1.5){e}{$e^{j 2 \pi \SI{30}{MHz} \cdot k/T_s}$}
        \pscircleop[operation=times](3.5,0){otimes}       
        \psblock(5,0){fft}{FFT}
        \pssignal(6.5,1.75){h}{$H[n]$}
        \pscircleop[operation=times](6.5,0){ftimes}
        \psblock(8,0){ifft}{iFFT}
        \psblock(10,0){buf}{buffer}
        \pscircleop[operation=times](11.5,0){btimes}
        \pssignal(13.25,0){out}{SMBV}
        \pssignal(13.25,-1){omul}{fixed}

        \psset{style=Arrow}
        \ncline{x}{avg}
        \ncline{avg}{otimes}
        \psset{doubleline=true}
        \nclist{ncline}{e,otimes,fft,ftimes,ifft,buf,btimes,out}
        \ncline{h}{ftimes}
        \ncangle[angleA=180,angleB=-90]{omul}{btimes}

        \fnode[doubleline=false,style=RoundCorners,style=Dash,linecolor=red,framesize=11 2.75](6.5,-0.5){box}
        \nput[offsetA=-4]{90}{box}{\textcolor{red}{DSP}}
    \end{pspicture}
    \caption{Overall datapath. Double lines indicate I/Q-data.}
    \label{fig:datapath}
\end{figure}
The signal processing task to solve consisted of the following steps (see
\Cref{fig:datapath}):
\begin{enumerate}
    \item Read data from ADC (\SI{16}{Bit} \SI{100}{MHz}, LVDS)
    \item Store data into Memory
    \item Averaging
    \item I/Q Demodulation (\SI{30}{MHz})
    \item Convolute values with configurable data
    \item Continuously output computed values (\SI{100}{MHz}, LVDS)
\end{enumerate}
Since data has to be continuously output, even while computing or reading
new data, the computed data is double buffered. Because of the high
data throughput needed this DSP algorithm has been implemented in an FPGA.\\
The ADC is an LTC2274 and has a single link LVDS as output with 8/10B line
coding. For synchronization the ADC sends the comma value K28.5
\cite{ltc2274}. Standard
SATA cabling has been chosen to connect the FPGA to the ADC with one pair
for clock from the FPGA and the other for data to the FPGA. To prevent the
need for a separate reset line, a small circuit asserts the SYNC pin of the
ADC in case of a clock pause. The ADC features an additional scrambling
circuit to minimise the noise caused by the digital interface.\\
The internal buffers were designed so that the FPGA is nearly fully utilized
which resulted in a buffer depth of \SI{48}{KiB}. Since such large FFT sizes
would fill the whole FPGA, overlap add has been used for the convolution with
possible FFT sizes ranging from 8 to 4096.\\
In order to read or write to the different memories these are mapped into the
address space of the integrated processor. The control pins are connected to
memory mapped registers. Processor and the rest of design work on different
clock domains. Because of that the control pins have synchronization circuits
and the memories can switch to the CPU clock domain. Since this clashes with
the need for continuous output, the active output buffer can only be read. If
one wants to send data to the output the inactive output buffer has to be
filled, followed by switching the output buffers.\\
To convert the signal back to analog form a R\&S SMBV100A is used. In order to
communicate the digital data to the SMBV the LVDS interface for
\emph{DIGITAL IQ IN/OUT} has been emulated.\\
The overall FPGA design can be seen in \Cref{fig:overallDesign} and is 
explained in more detail in \Cref{sec:hardware}.\\
For the rest of the document the following formating is used:
\begin{itemize}
    \item \signal{signal}
    \item \module{module}
    \item \clk{clock}
\end{itemize}
\begin{figure*}[p]
    \centering
    \begin{pspicture}(0,0)(15,20)
        \SpecialCoor
        \rput[tl](13.5,20){\psframebox{\shortstack[l]{%
            \legendline{blue}{1pt} data\\
            \legendline{red}{1pt} ctrl\\
            \legendline{green}{1pt} memory bus\\
            \legendline{magenta}{1pt} external\\
            \legendline{yellow}{4pt} sample\_clk\\
            \legendline{magenta}{4pt} core\_clk\\
            \legendline{cyan}{4pt} cpu\_clk}}
        }

        \pspolygon*[linecolor=yellow,opacity=0.4,linearc=0.2](2.7,18.9)(13.1,18.9)(13.1,13.4)(2.7,13.4)
        \pspolygon*[linecolor=yellow,opacity=0.4,linearc=0.2](2.7,3.6)(11.3,3.6)(11.3,0.3)(2.7,0.3)
        \pspolygon*[linecolor=magenta,opacity=0.4,linearc=0.2](2.7,13.4)(11.4,13.4)(11.4,3.6)(2.7,3.6)
        \pspolygon*[linecolor=cyan,opacity=0.4,linearc=0.1](0.15,15.05)(0.85,15.05)(0.85,7.75)(0.15,7.75)

        \rput[tr](13.3,19.9){top}
        \psframe(0,0)(13.4,20)
        \rput(1.7,0.2){%
            \rput[tr](11.4,19.1){main}
            \psframe(0,0)(11.5,19.2)
            \rput(1.1,13.3){%
                \psframe(0,0)(10.2,5.3)
                \rput[tr](10,5.2){inbuf}
                \pnode(0,0.2){inbufCtrl}
                \pnode(0,4.15){inbufDataIn}
                \pnode(0,2.85){inbufDataOut}
                \pnode(3.9,2){inbufDataMux}
                \pnode(1.9,1.4){averageMemCtrl}
                \rput(0,1.4){%
                    \rput[tr](3.4,3.3){average\_mem}
                    \psframe(0.2,0)(3.5,3.4)
                    \rput(0.6,0.4){%
                        \rput(1.7,1.7){\rnode{inMem}{\psframebox[fillstyle=solid]{mem}}}
                        \rput[t](2,1){\circlenode[fillstyle=solid]{inAdd}{$+$}}
                        \pnode(! \psGetNodeCenter{inAdd} 2.5 inAdd.y 0.25 add){averageAddLoop}
                        \pnode(! \psGetNodeCenter{inAdd} 2.5 inAdd.y 0.25 sub){averageAddIn}
                        \pnode(1,0.5){averageMuxOut}
                        \pnode(2.5,0.2){averageDataSplit}
                        \pnode(! \psGetNodeCenter{inAdd} 1.2 inAdd.y){averageMuxInAdd}
                        \pnode(1.2,0.2){averageMuxInData}
                        \ncline[linecolor=blue]{inAdd}{averageAddIn}
                        \ncline[linecolor=blue]{inAdd}{averageMuxInAdd}
                        \ncline[linecolor=green]{inAdd}{averageAddLoop}
                        \ncangle[linecolor=green, angleA=90, angleB=-90, offsetB=-1pt]{averageAddLoop}{inMem}
                        \ncloop[linecolor=green, armA=0.2, angleA=90, offsetA=1pt, angleB=180, loopsize=-1]{inMem}{averageMuxOut}
                        \ncline[linecolor=blue]{averageAddIn}{averageDataSplit}
                        \ncline[linecolor=blue]{averageDataSplit}{averageMuxInData}
                        \psdot[linecolor=blue](averageDataSplit)
                        \pnode(0.1,1.05){inMuxOut}
                        \pnode(0.1,2.35){inMuxIn}
                        \ncangle[linecolor=green, angleB=-90, offsetB=1pt, armB=0]{inMuxOut}{inMem}
                        \ncangle[linecolor=green, angleB=90, offsetB=1pt, armB=0]{inMuxIn}{inMem}
                        \ncline[linecolor=green]{inMuxOut}{inbufDataOut}
                        \ncline[linecolor=green]{inMuxIn}{inbufDataIn}
                        \rput(1,0){\pspolygon[fillcolor=white,fillstyle=solid](0,0.15)(0.2,0)(0.2,1)(0,0.85)}
                        \rput(0,0.7){\pspolygon[fillcolor=white,fillstyle=solid](0,0.15)(0.2,0)(0.2,0.7)(0,0.55)}
                        \rput{180}(0.2,2.7){\pspolygon[fillcolor=white,fillstyle=solid](0,0.15)(0.2,0)(0.2,0.7)(0,0.55)}
                    }
                }
                \rput(4.2,2){%
                    \psframe(0,-0.6)(5.8,0.9)
                    \rput[tr](5.7,0.8){receiver}
                    \multido{\i=2+-1,\n=0.1+-0.1}{3}{%
                        \pnode(!\n\space 3.5 add 5){GTXExt\i}
                        \rput(\n,\n){%
                            {
                                \psset{fillstyle=solid}
                                \rput[l](0.3,0){\rnode{descramble\i}{\psframebox{descramble}}}
                                \rput(3.5,0){\rnode{GTX\i}{\psframebox{GTX}}}
                                \rput(5,0){\rnode{align\i}{\psframebox{align}}}
                            }
                            \pnode(!\n\space neg -0.2 add 0){inbufDataMUX\i}
                            \ncline[linecolor=blue]{GTX\i}{descramble\i}
                            \ncline[linecolor=blue]{descramble\i}{inbufDataMUX\i}
                            \psset{linecolor=red}
                            \ncline{align\i}{GTX\i}
                            \ncangle[angleA=-90, armB=0]{align\i}{inbufCtrl}
                            \ncangle[angleA=-90, armB=0]{GTX\i}{inbufCtrl}
                            \ncangle[angleA=-90, armB=0]{descramble\i}{inbufCtrl}
                        }
                        \ncline[linecolor=magenta]{GTX\i}{GTXExt\i}
                    }
                }
                {
                    \psset{fillstyle=solid}
                    \rput[l](2.1,0.7){\rnode{wallclk}{\psframebox{wallclk}}}
                    \rput[r](1.6,0.7){\rnode{trigger}{\psframebox{trigger}}}
                }
                \ncline[linecolor=blue]{inbufDataMux}{averageDataSplit}
                {
                    \psset{linecolor=red}
                    \ncangle[angleA=-90, angleB=180, armB=0, offsetA=2pt]{averageMemCtrl}{wallclk}
                    \ncangle[angleA=-90, armB=0, offsetA=-2pt]{averageMemCtrl}{trigger}
                    \ncangle[angleA=-90, armB=0]{averageMemCtrl}{inbufCtrl}
                    \ncangle[angleA=-90, armB=0]{trigger}{inbufCtrl}
                    \ncangle[angleA=-90, armB=0]{wallclk}{inbufCtrl}
                    \ncangle[angleA=-90, armB=0]{inbufDataMux}{inbufCtrl}
                }
                \rput(3.8,1.65){\pspolygon[fillcolor=white,fillstyle=solid](0,0.15)(0.2,0)(0.2,0.7)(0,0.55)}
            }
            \rput(1.1,3.5){%
                \rput[tr](8.4,9.5){core}
                \psframe(0,0)(8.5,9.6)
                \rput(2,0.2){%
                    \rput[tr](6.2,8.8){overlap\_add}
                    \psframe(0,0)(6.3,8.9)
                    \rput(3,3.2){%
                        \rput[tr](3,5){fftncmul}
                        \psframe(0,0)(3.1,5.1)
                        \rput(0.2,3){%
                            \rput[tr](2.6,1.4){wave}
                            \psframe(0,-0.7)(2.7,1.5)
                            \rput(0.5,0){\circlenode[fillstyle=solid]{iqMulQ}{$\times$}}
                            \rput(2.2,0){\circlenode[fillstyle=solid]{iqMulI}{$\times$}}
                            \pnode(! \psGetNodeCenter{iqMulQ} iqMulQ.x 0.25 sub iqMulQ.y 0.5 add){iqMulQA}
                            \dotnode[linecolor=green](! \psGetNodeCenter{iqMulQ} iqMulQ.x 0.25 sub 1){iqMulInSplit}
                            \pnode(! \psGetNodeCenter{iqMulQ} iqMulQ.x 0.25 add iqMulQ.y 0.5 add){iqMulQB}
                            \pnode(! \psGetNodeCenter{iqMulI} iqMulI.x 0.25 sub iqMulI.y 0.5 add){iqMulIA}
                            \pnode(! \psGetNodeCenter{iqMulI} iqMulI.x 0.25 add iqMulI.y 0.5 add){iqMulIB}
                            \pnode(! \psGetNodeCenter{iqMulI} 1.35 iqMulI.y 0.5 add){iq}
                            \pnode(1.35,-0.5){iqMulOutSplit}
                            \ncline[linecolor=blue]{iq}{iqMulIA}
                            \ncline[linecolor=blue]{iqMulIA}{iqMulI}
                            \ncline[linecolor=blue]{iq}{iqMulQB}
                            \ncline[linecolor=blue]{iqMulQB}{iqMulQ}
                            \ncline[linecolor=green]{iqMulIB}{iqMulI}
                            \ncline[linecolor=green]{iqMulQA}{iqMulQ}
                            \ncline[linecolor=green]{iqMulQA}{iqMulInSplit}
                            \ncangle[linecolor=green, angleB=90]{iqMulInSplit}{iqMulIB}
                            \ncangle[linecolor=blue, angleA=-90, angleB=180]{iqMulQ}{iqMulOutSplit}
                            \ncangle[linecolor=blue, angleA=-90]{iqMulI}{iqMulOutSplit}
                            \rput(! \psGetNodeCenter{iqMulI} 1 iqMulI.y 0.15 add){%
                                \psframe[fillstyle=solid](0,0)(0.7,0.7)
                                \psline(0.01,0.01)(.69,.69)
                                \rput(0.21,0.525){\parametricplot[linewidth=.7pt, plotpoints=100]{-1}{1}{0.175 t mul t 180 mul sin -0.0875 mul}}
                                \rput(0.49,0.175){\parametricplot[linewidth=.7pt, plotpoints=100]{-1}{1}{0.175 t mul t 180 mul cos -0.0875 mul}}
                            }
                        }
                        \rput(1,1){\circlenode[fillstyle=solid]{fftncmulMul}{$\times$}}
                        \pnode(! \psGetNodeCenter{fftncmulMul} fftncmulMul.x 0.25 sub fftncmulMul.y 0.5 add){fftncmulMulA}
                        \pnode(! \psGetNodeCenter{fftncmulMul} fftncmulMul.x 0.25 add fftncmulMul.y 0.5 add){fftncmulMulB}
                        \ncline[linecolor=blue]{fftncmulMul}{fftncmulMulA}
                        \ncline[linecolor=green]{fftncmulMul}{fftncmulMulB}
                        \pnode(1.05,0){fftncmulCtrl}
                    }
                    \rput(3,0.2){%
                        \rput[tr](2,2.5){ifftnadd}
                        \psframe(0,0)(2.1,2.6)
                        \rput(1,1){\circlenode[fillstyle=solid]{ifftnaddAdd}{$+$}}
                        \pnode(! \psGetNodeCenter{ifftnaddAdd} ifftnaddAdd.x 0.25 sub ifftnaddAdd.y 0.5 add){ifftnaddAddA}
                        \pnode(! \psGetNodeCenter{ifftnaddAdd} ifftnaddAdd.x 0.25 add ifftnaddAdd.y 0.5 add){ifftnaddAddB}
                        \ncline[linecolor=blue]{ifftnaddAdd}{ifftnaddAddA}
                        \ncline[linecolor=green]{ifftnaddAdd}{ifftnaddAddB}
                        \pnode(1.05,2.6){ifftnaddCtrl}
                    }
                    \rput(2.4,3){\rnode{fft}{\psframebox[fillstyle=solid]{fft}}}
                    \rput[l](0.2,2.5){\rnode{scratch}{\psframebox[fillstyle=solid]{scratch}}}
                    \pnode(! \psGetNodeCenter{ifftnaddAdd} 3.2 ifftnaddAdd.y 0.5 0.2 add add){fftOutSplit}
                    \psdot[linecolor=blue](fftOutSplit)
                    \ncangle[linecolor=blue, angleA=-90, angleB=180, armB=0.2]{fft}{fftOutSplit}
                    \ncangle[linecolor=blue, angleA=-90, angleB=90, armB=0.2]{fftOutSplit}{fftncmulMulA}
                    \ncangle[linecolor=blue, angleA=-90, angleB=90, armB=0.2]{fftOutSplit}{ifftnaddAddA}
                    \pnode(2.9,3){fftCtrlSplit}
                    \pnode(2.4,4.1){fftMux}
                    \ncangle[linecolor=red, angleB=-90, armB=0]{fftCtrlSplit}{fftncmulCtrl}
                    \ncangle[linecolor=red, angleB=90, armB=0]{fftCtrlSplit}{ifftnaddCtrl}
                    \ncangle[linecolor=red, angleA=90]{fftCtrlSplit}{fftMux}
                    \ncline[linecolor=red]{fft}{fftCtrlSplit}
                    \ncangle[linecolor=green, angleA=90, angleB=-90]{scratch}{fftncmulMul}
                    \ncangle[linecolor=blue, angleA=-90, angleB=90, offsetB=0.15, armB=1.2]{iqMulOutSplit}{fftMux}
                    \ncline[linecolor=blue]{fftMux}{fft}
                    \pnode(1.85,2.05){scratchSplit}
                    \ncangle[linecolor=green, angleA=-90, angleB=180]{scratch}{scratchSplit}
                    \ncangle[linecolor=green, angleA=90, angleB=90, offsetB=-0.15]{scratchSplit}{fftMux}
                    \psdot[linecolor=green](scratchSplit)
                    \ncangle[linecolor=green, angleB=90]{scratchSplit}{ifftnaddAddB}
                    \rput{90}(2.75,4){\pspolygon[fillcolor=white,fillstyle=solid](0,0.15)(0.2,0)(0.2,0.7)(0,0.55)}
                }
                \rput(1,5.7){\rnode{H}{\psframebox[fillstyle=solid]{H}}}
                \ncangle[linecolor=green, angleA=-90, angleB=90, offsetA=1pt]{H}{fftncmulMulB}
                \pnode(0,8){coreCtrl}
                \pnode(2,8){overlapAddCtrl}
                \pnode(0,7.4){coreIn}
                \pnode(0,6.2){coreHIn}
                \pnode(0,5.25){coreHOut}
                \pnode(0,0.6){coreOut}
                \ncline[linecolor=red]{coreCtrl}{overlapAddCtrl}
                \ncline[linecolor=green]{coreIn}{iqMulInSplit}
                \ncangle[linecolor=green, angleB=90, armB=0]{coreHIn}{H}
                \ncangle[linecolor=green, angleB=-90, armB=0, offsetB=1pt]{coreHOut}{H}
                \ncangle[linecolor=green, angleB=-90, armB=0]{coreOut}{ifftnaddAdd}
            }
            \rput(1.1,0.2){%
                \psframe(0,0)(8.4,3.1)
                \rput[tr](8.3,3.0){outbuf}
                \rput(1,2){\rnode{outMem0}{\psframebox[fillstyle=solid]{mem\_0}}}
                \rput(3,2){\rnode{outMem1}{\psframebox[fillstyle=solid]{mem\_1}}}
                \pnode(0,0.2){outCtrl}
                \pnode(0,0.4){outDataOut}
                \pnode(0,2.5){outDataIn}
                \pnode(1,2.5){outDataSplit}
                \pnode(2,0.8){outMemMuxOut}
                \pnode(4.2,1.45){outMemMuxData}
                \ncangle[linecolor=green, angleA=-90, angleB=90, offsetB=-0.1, offsetA=-0.1]{outMem0}{outMemMuxOut}
                \ncangle[linecolor=green, angleA=-90, angleB=90, offsetB=0.1, offsetA=-0.1]{outMem1}{outMemMuxOut}
                \ncangle[linecolor=green, angleA=-90, angleB=180, offsetB=-0.1, offsetA=0.1]{outMem0}{outMemMuxData}
                \ncangle[linecolor=green, angleA=-90, angleB=180, offsetB=0.1, offsetA=0.1]{outMem1}{outMemMuxData}
                \rput(5.2,1.45){\circlenode[fillstyle=solid]{outMul}{$\times$}}
                \rput[l](6,1.45){\rnode{transmitter}{\psframebox[fillstyle=solid]{transmitter}}}
                \ncline[linecolor=blue]{outMemMuxData}{outMul}
                \ncangle[linecolor=red, angleB=-90, armA=0, armB=0]{outMemMuxOut}{outMemMuxData} 
                \ncangle[linecolor=red, angleB=-90, armA=0, armB=0]{outCtrl}{outMemMuxData}
                \ncangle[linecolor=red, angleB=-90, armA=0, armB=0]{outCtrl}{outMul}
                \ncangle[linecolor=red, angleB=-90, armA=0, armB=0]{outCtrl}{transmitter}
                \ncline[linecolor=green]{outDataIn}{outDataSplit}
                \ncline[linecolor=green]{outDataSplit}{outMem0}
                \psdot[linecolor=green](outDataSplit)
                \ncangle[linecolor=green, angleB=90, armA=0, armB=0]{outDataSplit}{outMem1}
                \ncangle[linecolor=green, angleB=-90, armA=0, armB=0]{outDataOut}{outMemMuxOut}
                \rput{90}(2.35,0.7){\pspolygon[fillcolor=white,fillstyle=solid](0,0.15)(0.2,0)(0.2,0.7)(0,0.55)}
                \rput{180}(4.3,1.8){\pspolygon[fillcolor=white,fillstyle=solid](0,0.15)(0.2,0)(0.2,0.7)(0,0.55)}
                \ncline[linecolor=blue]{outMul}{transmitter}
            }
            \pnode(0.7,16.15){mainMuxInInSplit}
            \psdot[linecolor=green](mainMuxInInSplit)
            \ncline[linecolor=green]{mainMuxInInSplit}{inbufDataOut}
            \ncangle[linecolor=green, angleA=90, angleB=180]{mainMuxInInSplit}{coreIn}
            \pnode(0.7,4.1){outbufMuxInCoreSplit}
            \psdot[linecolor=green](outbufMuxInCoreSplit)
            \pnode(0.4,3.1){outbufMuxInExt}
            \pnode(0.7,3.1){outbufMuxInCore}
            \pnode(0.55,2.9){outbufMuxOut}
            \ncangle[linecolor=green, angleA=90, angleB=180]{outbufMuxOut}{outDataIn}
            \ncline[linecolor=green]{outbufMuxInCore}{outbufMuxInCoreSplit}
            \ncline[linecolor=green]{outbufMuxInCoreSplit}{coreOut}
            \pnode(0,17.45){mainInbufIn}
            \pnode(0,16.15){mainInbufOut}
            \pnode(0,13.5){mainInbufCtrl}
            \pnode(0,11.5){mainCoreCtrl}
            \pnode(0,9.7){mainCoreHIn}
            \pnode(0,8.75){mainCoreHOut}
            \pnode(0,4.1){mainCoreOut}
            \pnode(0,3.3){mainOutbufIn}
            \pnode(0,0.6){mainOutbufOut}
            \pnode(0,0.4){mainOutbufCtrl}
            \ncline[linecolor=green]{mainInbufIn}{inbufDataIn}
            \ncline[linecolor=green]{mainInbufOut}{mainMuxInInSplit}
            \ncline[linecolor=red]{mainInbufCtrl}{inbufCtrl}
            \ncline[linecolor=red]{mainCoreCtrl}{coreCtrl}
            \ncline[linecolor=green]{mainCoreHIn}{coreHIn}
            \ncline[linecolor=green]{mainCoreHOut}{coreHOut}
            \ncline[linecolor=green]{mainCoreOut}{outbufMuxInCoreSplit}
            \ncangle[linecolor=green, angleB=90, armB=0]{mainOutbufIn}{outbufMuxInExt}
            \ncline[linecolor=green]{mainOutbufOut}{outDataOut}
            \ncline[linecolor=red]{mainOutbufCtrl}{outCtrl}
            \rput{90}(0.9,2.9){\pspolygon[fillcolor=white,fillstyle=solid](0,0.15)(0.2,0)(0.2,0.7)(0,0.55)}
        }
        \pnode(1.3,8.75){memMuxOutbuf}
        \pnode(1.3,8.85){memMuxCore}
        \pnode(1.3,8.95){memMuxCoreH}
        \pnode(1.3,9.05){memMuxInbuf}
        \ncangles[linecolor=green, angleA=180, armA=0.2, armB=0]{mainOutbufOut}{memMuxOutbuf}
        \ncangles[linecolor=green, angleA=180, armA=0.1, armB=0]{mainCoreOut}{memMuxCore}
        \ncangles[linecolor=green, angleA=180, armA=0.1, armB=0]{mainInbufOut}{memMuxInbuf}
        \ncline[linecolor=green]{mainCoreHOut}{memMuxCoreH}
        \pnode(1.1,8.9){memMuxOut}
        \rput[c]{90}(0.5,13.7){\rnode{procRegister}{\psframebox[fillstyle=solid]{proc\_register}}}
        \rput[c]{90}(0.5,8.9){\rnode{procMem}{\psframebox[fillstyle=solid]{proc\_mem}}}
        \rput[c]{90}(0.5,11.2){\rnode{cpu}{\psframebox[fillstyle=solid]{cpu}}}
        \ncline[linecolor=green, offsetA=0.05, offsetB=0.05]{memMuxOut}{procMem}
        \rput(1.1,8.55){\pspolygon[fillcolor=white,fillstyle=solid](0,0.15)(0.2,0)(0.2,0.7)(0,0.55)}
        \pnode(1.0,8.95){memBusSplit}
        \psdot[linecolor=green](memBusSplit)
        \ncline[linecolor=green, offsetB=-0.02]{memBusSplit}{procMem}
        \ncangle[linecolor=green, angleB=-90]{mainOutbufIn}{memBusSplit}
        \pnode(0.9,13.7){ctrlBusSplit}
        \psdot[linecolor=red](ctrlBusSplit)
        \pnode(0.9,11.7){ctrlBusSplitCore}
        \psdot[linecolor=red](ctrlBusSplitCore)
        \pnode(1.0,9.9){memBusSplitCoreH}
        \psdot[linecolor=green](memBusSplitCoreH)
        \ncangle[linecolor=green, angleB=90]{mainInbufIn}{memBusSplitCoreH}
        \ncline[linecolor=green]{memBusSplitCoreH}{memBusSplit}
        \ncline[linecolor=green]{mainCoreHIn}{memBusSplitCoreH}
        \ncangle[linecolor=red, angleB=90]{mainOutbufCtrl}{ctrlBusSplitCore}
        \ncline[linecolor=red]{mainCoreCtrl}{ctrlBusSplitCore}
        \ncline[linecolor=red]{mainInbufCtrl}{ctrlBusSplit}
        \ncline[linecolor=red]{ctrlBusSplitCore}{ctrlBusSplit}
        \ncline[linecolor=red]{procRegister}{ctrlBusSplit}
        \ncline[linecolor=green]{procRegister}{cpu}
        \ncline[linecolor=green]{procMem}{cpu}

        \pnode(-0.5,11.2){cpuExt}
        \ncline[linecolor=magenta]{cpu}{cpuExt}
        \pnode(13.9,1.85){transmitterExt}
        \ncline[linecolor=magenta]{transmitter}{transmitterExt}
    \end{pspicture}
    \caption{Overall Functional Design with the VHDL module names.}
    \label{fig:overallDesign}
\end{figure*}

\section{Hardware}
\label{sec:hardware}
This project uses ISE 10.1 to implement the Hardware on a Virtex 5 (XC5VFX70T).
The whole project is structured into the modules \module{inbuf}
(\Cref{sec:inbuf}), \module{core} (\Cref{sec:core}) and \module{outbuf}
(\Cref{sec:outbuf}). These modules are split into several clock domains as can
be seen in \Cref{fig:overallDesign}:
\begin{description}
    \item[\clk{sample\_clock}:] \hfill \\
        Clock recovered from LVDS (GTX \cite[p. 179ff]{gtx}). \SI{100}{MHz}.
    \item[\clk{core\_clock}:] \hfill \\
        This clock is only used for the DSP core, to be able
        to speed up the calculations. Due to synthesis constraints it is
        currently the same as \clk{sample\_clock}. \SI{100}{MHz}.
    \item[\clk{cpu\_clock}] \hfill \\
        Internal CPU bus clock. \SI{100}{Mhz}; not aligned to
        \clk{sample\_clock}.
\end{description}
Since \clk{sample\_clock} and \clk{cpu\_clock} are not related, all affected memory
clocks are sourced from clock muxes. Therefore it is possible to do only
one of the following activities at a time:
\begin{itemize}
    \item Capture data from ADC.
    \item Convolute.
    \item Read/Write data with the CPU.
\end{itemize}
This is guarded by \signal{core\_busy} and \signal{mem\_req} (see
\Cref{sec:register}, \Cref{reg5}).\\
Xilinx wizards and inferring from VHDL have been avoided for the memories,
because those achieved suboptimal results, which could be cause by the
high FPGA utilization.\\
The whole project can be downloaded from
\url{https://github.com/notti/bak-hardware}.
\subsection{Module \module{inbuf}}
\label{sec:inbuf}
This module consists of the following submodules:
\begin{description}
    \item[\module{align}:] \hfill \\
        Simple state machine for generating the clock pauses to sync with
        the ADC. The timings can be configured with generics.
    \item[\module{GTX}:] \hfill \\
        GTX transceiver \cite{gtx} configured by wizard\cite{gtx_wizard}. This component has to be
        replaced in case the FPGA changes. Because of the abundant number of
        signals and generics instantiating this module by hand is not advised.
        Runtime configuration pins are exposed via \Cref{reg0}.
    \item[\module{descramble}:] \hfill \\
        Descrambler modelled after \cite[p. 28]{ltc2274}.
    \item[\module{trigger}:] \hfill \\
        Trigger for data capture. Can use either external trigger (Signal
        \signal{inbuf\_trigger} pin~AN33) or internal trigger which triggers
        on position 0 of a frame. If no frame has been captured yet, the
        internal trigger can be fired with \signal{trig\_int} (see \Cref{reg1}).
    \item[\module{wallclk}:] \hfill \\
        Generates frame position, frame trigger and start index for \module{wave}
        (I/Q demodulation).
    \item[\module{average\_mem}:] \hfill \\
        Averaging memory, that can average over 1, 2, 4, 8 samples. Averaging
        is achieved by adding the samples from consecutive runs and shifting
        the values on memory read.
\end{description}
Since there are three GTX transceivers \cite{gtx} easily accessible on the ML507 board
(two SATA connectors and one with four SMA headers; see \cite{ml507}), all
three have been instantiated. A multiplexer allows changing the input during
runtime (see \Cref{reg0}).
\subsection{Module \module{core}}
\label{sec:core}
\module{core} is mainly a wrapper for \module{overlap\_add} which does all
the calculations. For the (i)FFT the Xilinx IP \cite{xilinx_fft} has been
used. Since integer values are used during every step of the calculation the
bit width increases, scaling with convergent rounding is used. To achieve a
faster computation the pipelined version was chosen. \module{wave} implements
the I/Q demodulation which is realized with a ROM consisting of the $\sin$
and $\cos$ values for 10 cycles (\SI{30}{MHz} sampled at \SI{100}{MHz}) and
multipliers.\\
\begin{figure}[t]
    \centering
    \begin{pspicture}(0,0)(7.5,6.9)
        \psset{braceWidth=.5pt,braceWidthInner=4pt,braceWidthOuter=4pt,nodesepB=-2pt,rot=-90}
        \psbrace[ref=B](3,5.1)(1,5.1){$L$}
        \psbrace[ref=B](3.5,5.7)(1,5.7){$nfft$}
        \psbrace[ref=B](5.5,6.3)(1,6.3){$n$}
        \psset{hatchsep=2pt,dotsep=1pt}
        \pspolygon(1,4.5)(5.5,4.5)(5.5,5)(1,5)
        \psline(3,4.5)(3,5)
        \psline(5,4.5)(5,5)
        \psline[linestyle=dashed](5.5,4.5)(7,4.5)(7,5)(5.5,5)

        \psline(3,3)(1,3)(1,3.5)(3,3.5)
        \psline[fillstyle=crosshatch](3,3)(3.5,3)(3.5,3.5)(3,3.5)

        \psline(5,2)(3,2)(3,2.5)(5,2.5)
        \psline[fillstyle=crosshatch](5,2)(5.5,2)(5.5,2.5)(5,2.5)

        \psline(7,1)(5,1)(5,1.5)(7,1.5)
        \psline[linestyle=dotted](7,1)(7.2,1)
        \psline[linestyle=dotted](7,1.5)(7.2,1.5)
        {
            \psset{linecolor=red,hatchcolor=red}
            \psline[linestyle=dotted](0.8,1)(1,1)
            \psline[linestyle=dotted](0.8,1.5)(1,1.5)
            \psline[fillstyle=crosshatch](1,1)(1.5,1)(1.5,1.5)(1,1.5)
        }

        \pspolygon(1,0)(5.5,0)(5.5,0.5)(1,0.5)

        \psset{dotsep=1pt,linestyle=dotted,linecolor=gray}
        \psline(1,3.5)(1,4.5)
        \psline(3.5,3.5)(3,4.5)

        \psline(1,0.5)(1,1)
        \psline(1,1.5)(1,3)

        \psline(1.5,0.5)(1.5,1)
        \psline(1.5,1.5)(1.5,3)

        \psline(3,0.5)(3,2)
        \psline(3,2.5)(3,3)

        \psline(3.5,0.5)(3.5,2)
        \psline(3.5,2.5)(3.5,3)

        \psline(5,0.5)(5,1)
        \psline(5,1.5)(5,2)

        \psline(5.5,0.5)(5.5,1)
        \psline(5.5,1.5)(5.5,2)

        \psline(5.5,2.5)(5.5,3.5)(5,4.5)

        \psline(7.25,4)(7,4.5)

        \rput(6.25,4.75){$000000$}

        \rput(3.25,2.75){$+$}
        \rput(5.25,1.75){$+$}
        \rput(1.25,2.25){$+$}

        \rput(0.25,4.75){$x[n]$}
        \rput(0.25,0.25){$y[n]$}
        \rput(0.25,3.25){$y_0[n]$}
        \rput(0.25,2.25){$y_1[n]$}
        \rput(0.25,1.25){$y_2[n]$}

    \end{pspicture}
    \caption{Overlap add algorithm. The red part is only needed for cirular
        convolution. $y_n[n] = \mathcal{F}^{-1}\left(\mathcal{F}\left(
        x[L\cdot n:(L+1)\cdot n], nfft \right) \cdot H[n] \right)$
        where $\mathcal{F}\left(x, n \right)$ denotes the zero extended FFT of $x$.}
    \label{fig:olapadd}
\end{figure}
Overlap add as described in \Cref{fig:olapadd} can be used to calculate the
linear or circular convolution. This algorithm needs the following steps:
\begin{enumerate}
    \item Zero extend the input signal $x[n]$ to a multiple of $L$.
    \item Take an $L$ sized block out of $x[n]$.
    \item Zero extend the block to a length of $nfft$.
    \item Fourier transform the block.
    \item Multiply the result with $H[n]$.
    \item Transform the result back.
    \item $y[n]$ is Steps 2-6 done with every block. The overlapping
        parts need to be added together.
    \item For cirular convolution the overlapping part of the last block has
        to be added to the beginning of $y[n]$
\end{enumerate}
In hardware the zero extending is achived by a multiplexer zeroing the inputs.
Another difference is that the result as calculated by the used FFT IP
\cite{xilinx_fft} is in bit reversed order. This does not matter for the iFFT,
because the result gets written to memory and there the order is not important.
After the FFT however the result has to be sorted so it can be used again as 
input for the FFT IP \cite{xilinx_fft}. For this the memory \module{scratch}
is used. Scratch is also filled during the iFFT with the current overlapping
block from the result, for easier calculation of the addition. The state machine
controlling the whole process is split into the two modules \module{fftncmul} and
\module{ifftnadd}. With this split it is possible to run both at the same time,
resulting in a higher trough put. \Cref{fig:olapadd_flow} describes how this
works.
\begin{figure}[bh]
    \centering
    \begin{pspicture}(9.2,6.8)
        \rput[bl](-0.2,0){
        {
            \psset{style=RoundCorners,style=Dash}
            \fnode[linecolor=red,framesize=5 1.3](4.6,5.75){fft}
            \nput{90}{fft}{\textcolor{red}{\module{fftncmul}}}
            \pspolygon[linecolor=purple](7.4,6.3)(9.2,6.3)(9.2,0.5)(3.8,0.5)(3.8,4.9)(7.4,4.9)
            \rput(6.5,0.25){\textcolor{purple}{\module{ifftnadd}}}
        }
        \psmatrix[colsep=.5cm,rowsep=.5cm,mnode=r]
        [mnode=oval] start & \psframebox{FFT} & & \psframebox{\shortstack{$\times H[n]$ $\rightarrow$\\ \module{scratch}}} & \psframebox{iFFT} \\
        & [mnode=dia] last & [mnode=p] & \psframebox{\shortstack{$y[n] \rightarrow$\\\module{scratch}}} & [mnode=dia] first \\
        & & [mnode=p] & \psframebox{\shortstack{$0 \rightarrow$\\\module{scratch}}} & \\
        & [mnode=dia] last & & \psframebox{\shortstack{$+$\module{scratch}\\$\rightarrow y[n]$}} & [mnode=r] \\
        & [mnode=oval] finished & & &
        \psset{arrows=->}
        \ncline{1,1}{1,2}
        \ncline{1,2}{1,4}\naput{wait}\nbput{FFT}
        \ncline{1,4}{1,5}
        \ncline{1,5}{2,5}
        \ncangle[angleA=-90]{2,5}{3,4}\naput[npos=0]{Yes}
        \ncline{2,5}{2,4}\nbput[npos=0]{No}
        \ncline[arrows=-]{2,4}{2,3}
        \ncline[arrows=-]{3,4}{3,3}
        \ncangle[angleA=90]{3,3}{2,2}
        \ncline{2,2}{1,2}\nbput[npos=0]{No}
        \ncangle[arrows=-,armB=0.8,angleA=-90,angleB=90]{3,3}{4,5}
        \ncline{4,5}{4,4}\nbput{wait}\naput{iFFT}
        \ncline{4,4}{4,2}
        \ncline{4,2}{5,2}\nbput[npos=0]{Yes}
        \endpsmatrix}
    \end{pspicture}
    \caption{Overlap add control/data flow.}
    \label{fig:olapadd_flow}
\end{figure}
The module \module{core} has the following control signals:
\begin{description}
    \item[\signal{start}] \hfill \\
        Starts the convolution. All the other signals are sampled on
        \signal{start} and can change during the rest of the time.
    \item[\signal{n}] \hfill \\
        Size of the fft in $\log_2\left( nfft \right)$.
    \item[\signal{scale\_sch}:] \hfill \\
        Scaling schedule for the FFT. See \cite{xilinx_fft}.
    \item[\signal{scale\_schi}:] \hfill \\
        Scaling schedule for the iFFT. See \cite{xilinx_fft}.
    \item[\signal{scale\_cmul}:] \hfill \\
        Scaling applied after the complex multiplication.
    \item[\signal{L}:] \hfill \\
        $L$
    \item[\signal{depth}:] \hfill \\
        $n$
    \item[\signal{iq}:] \hfill \\
        Switches I/Q-demodulation on/off.
    \item[\signal{circular}:] \hfill \\
        Selects between circular and linear convolution.
    \item[\signal{ov\_*}:] \hfill \\
        After a complete run these indicate if an overflow happened during
        \begin{itemize}
            \item FFT
            \item iFFT
            \item Complex multiplication
        \end{itemize}
    \item[\signal{busy}:] \hfill \\
        High during run.
    \item[\signal{done}:] \hfill \\
        Pulses after complete run.
    \item[\signal{wave\_index}:] \hfill \\
        Wave index of the first value. Needed for the I/Q-demodulation.
\end{description}

\subsection{Module \module{outbuf}}
\label{sec:outbuf}
\begin{figure*}[b]
    \centering
    \begin{tikztimingtable}
        \clk{clk} & c11{7c}c  \\
        \clk{ckm} & c22{3.5c}c \\
        \clk{ckh} & c77{c}c \\
        \signal{outdata\_long}(0) & x2{14x}{}3{14d}7dd\\
        \signal{outdata\_long}(1) & x3{14x}{}2{14d}7dd \\
        \signal{outdata\_long}(2) & x{}4{14d}{}14d7dd \\
        \signal{outdata\_long}(3) & x14x{}4{14d}{}7dd \\
        \signal{outdata\_short} & [font=\scriptsize] x7d{(0)(6:3)}7d{(0)(2:0) (1)(6)}7d{(1)(5:2)}7d{(1)(1:0) (2)(6:5)}7d{(2)(4:1)}7d{(2)(0) (3)(6:4)}7d{(3)(3:0)}7d{(0)(6:3)}7d{(0)(2:0) (1)(6)}7d{(1)(5:2)}7d{(1)(1:0) (2)(6:5)}{}d \\
        \signal{out\_en} & l49{l}28{h}h \\
    \end{tikztimingtable}
    \caption{LVDS Output generation timing. {\scriptsize (0)(2:0) (1)(6)} in \signal{outdata\_short} means {\ttfamily outdata\_long(0)(2 downto 0) \& outdata\_long(1)(6 downto 6)}.}
    \label{fig:trans}
\end{figure*}
This module consists of the two main parts \module{transmitter} and the two
memory buffers for holding the result. The active buffer can be switched with
a pulse to \signal{toggle\_buf}. Switching gets deferred until the address
counter reaches zero. A successful switch is reported with a pulse to
\signal{toggled}. This module is also able to synchronize the output cycle
to the input cycle with asserting \signal{resync}. The data gets multiplied by
\signal{muli}/\signal{mulq} before transmission. For handling overflows, which
would be reported by \signal{ovfl}, saturation (\signal{sat}) or shift
(\signal{shift}) can be used.\\
For communication with the SMBV the interface desribed by \cite{fsq_b17} has
been implemented in VHDL. To be able to communicate with the R\&S SMBV100A the
pinout mentioned in \cite[p. 9]{fsq_b17} has been used. One undocumented
change has to be made: In order for the SMBV100A to be
able to detect connected Hardware the S\_CLK pin has to be connected to GND.
According to \cite[p. 10]{fsq_b17}, the different signals are assigned to
\signal{e1} (D24 - D47) and \signal{e2} (D0-D23). The triggers and markers are
hardcoded to zero, because the SMBV100A is not able to use those signals from
\emph{DIGITAL I/Q IN/OUT}. Since the data is always valid and enabled, those two
signals are hard coded to one. \cite[p. 10]{fsq_b17} also states that the LVDS
link is similar to the one implemented by a TI DS90CR485. Because of this the module
\module{transmitter} is a clean room VHDL implementation written according to
the specification in \cite{ds90cr485}. Because at the time of the implementation,
the settings of the receiver were unknown, the whole feature set of the
transmitter with every control signal has been implemented. The balance value
of a value is calculated with a pre filled ROM, that has been generated with
the script \emph{balance\_table.py}. Since there are no suitable GTX
transceivers \cite{gtx} available, SelectIO\cite[p. 270ff]{virtex5} is used
for the transmitters. In order to create the LVDS outputs with the same
characteristics, OSERDES \cite[p. 370ff]{virtex5} elements in
DDR mode have been used. This results in the high clock frequencies of
\SI{350}{MHz} (=7 Bit per \SI{100}{MHz} Cycle/2 Bit per Cycle, \clk{ckh}) and
\SI{175}{MHz} (\clk{ckm}). To achieve the timing goals and correct ordering
of the output signals, the timing in \Cref{fig:trans} was implemented and the
last registers and multiplexers in front of the OSERDES
\cite[p. 370ff]{virtex5} were assigned fixed locations in the floor plan
(see \Cref{sec:synthesis}).

\subsection{Module \module{proc\_register}}
\label{sec:register}

The control communication between the processor and hardware is handled by this
module. Since there are several clock domains involved this module also handles
the clock synchronization if needed. This is done with the module \module{flag}
which is a simple two stage synchronizer with two flip flops. Since this likely
causes timing issues during synthesis, this module also sets the needed
attributes:
\begin{description}
    \item[\protect{TIG = TRUE\cite[p. 302]{constraints}:}] \hfill \\
        Ignore timing on this signal.
    \item[\protect{IOB = FALSE\cite[p. 146]{constraints}:}] \hfill \\
        Prevents packing this into an IO block. This is needed,
        because an IO block contains only one flip flop and in order to ensure
        proper synchronization the two flip flops shouldn't have long
        connections in between.
    \item[\protect{ASYNC\_REG = TRUE\cite[p. 87]{constraints}:}] \hfill \\
        This constraint disables propagation of 'X' during timing simulation.
    \item[\protect{SHIFT\_EXTRACT = FALSE\cite[p. 385]{xst}:}] \hfill \\
        This prevents inferring a dedicated shift register.
    \item[\protect{HBLKNM = name\cite[p. 134]{constraints}:}] \hfill \\
        This assigns a hierarchical block name to both flip flops to ensure
        both end up in the same slice.
\end{description}
The mentioned module can only synchronize slow changing single signal logic
values. For
synchronizing a single pulse \module{toggle} has to be used. In order to
synchronize a value consisting of several signals, \module{value} has to
be used. This module prevents the assertion of \signal{fpga2bus\_wrack}
to guarantee the value doesn't change during synchronization.\\
Input and output ports of \module{proc\_register} are sorted by
needed synchronization/clock and the source contains a table which describes
every signal.\\
\module{proc\_register} also generates the interrupt register. This register
is sampled by the user logic in the processor module and generates an
interrupt on the rising edge.\\
\Crefrange{reg0}{intr} contain a list of all the registers and the meaning of
every bit therein.
\subsection{Processor}
\label{sec:processor}
In order to be able to communicate the internal PowerPC 440 \cite{ppc} is used.
This block is incorporated into the design by instantiating the module
\module{processor}, which is generated by EDK, in the top level module.
The following settings are needed to be able to run linux:
\begin{itemize}
    \item Enabled cache setup.
    \item One UART. XPS\_UART16550. Use interrupt.
    \item GPIO controllers if needed. Interrupts are not supported by the
        linux driver.
    \item Hard ethernet MAC. Use interrupt.
    \item DDR2 SDRAM.
    \item SysACE\_CompactFlash. Use interrupt.
    \item XPS\_BRAM\_IF\_CNTLR\_1. 8 KB. This needs to be at the highest
        address.
    \item Cache enabled for the memories.
\end{itemize}
Communication with the rest of the design is handled by the IP
\module{proc2fpga}. To
keep the everything in one place and prevent switching between two projects
(ISE and EDK), the this module is kept very basic.
\module{proc2fpga} is created with the \emph{Create or Import Peripheral\ldots}
wizard in the EDK. Settings used are:
\begin{itemize}
    \item Processor Local Bus.
    \item Software reset.
    \item Interrupt control.
    \item User logic software register.
    \item User logic memory space.
    \item Burst and cache-line support. This is needed, so the memories can
        be directly accessed by linux. Data width: 32 Bit. Write buffer depth:
        32.
    \item Use Device ISC. 16 Interrupts, edge detect.
    \item 6 registers.
    \item 4 memory address ranges.
\end{itemize}
\lstset{language=ucf}
In the generated IP, only very basic logic has been added. This logic only
converts between the different bit orderings and exposes all signals, which
were generated by the wizard, to the rest of the design. To be able to route
these signals outside the processor design, these have to be annotated in the
mpd description file with \lstinline+BUFFER_TYPE = none+. Without this modification
EDK instantiates I/OBUFs for these signals, which will generate errors,
since those signals won't be connected to a pad of the FPGA. After generating
this, the final module is reimported back into the EDK and added to the
design.\\
The exposed signals are listed in \cite{slave_burst}. \cite[p. 14]{slave_burst}
mentions that \signal{IP2Bus\_Error} is only sampled during acknowledge, which
seems to be wrong. Asserting this signal outside of a read or write cycle
causes random \emph{SIGBUS} faults.\\
In order to run an operating system on the generated processor system, a
description of the used modules and settings like memory addresses needs to be
generated. The description format used is device tree\cite{dt}. For EDK a tool
is available that can generate the description. Instructions and a download
link is provided at \url{http://www.wiki.xilinx.com/Build+Device+Tree+Blob}.
The full project already has this tool included.
\subsection{Synthesis}
\label{sec:synthesis}
\begin{lstlisting}[float,language=ucf,caption={Floor plan example},label=lst:floorplan,basicstyle=\ttfamily\tiny]
INST "*/transmitter_i/outdata_short6_0" LOC=SLICE_X2Y78;
INST "*/transmitter_i/outdata_short6_1" LOC=SLICE_X2Y78;
INST "*/transmitter_i/mux24_*" LOC=SLICE_X2Y78;
INST "*/transmitter_i/mux25_*" LOC=SLICE_X2Y78;
INST "*/transmitter_i/outdata_short6_2" LOC=SLICE_X3Y78;
INST "*/transmitter_i/outdata_short6_3" LOC=SLICE_X3Y78;
INST "*/transmitter_i/mux26_*" LOC=SLICE_X3Y78;
INST "*/transmitter_i/mux27_*" LOC=SLICE_X3Y78;

#####

AREA_GROUP "inbuf" RANGE=SLICE_X54Y70:SLICE_X67Y89;
AREA_GROUP "inbuf" RANGE=DSP48_X1Y28:DSP48_X1Y35;
AREA_GROUP "inbuf" RANGE=RAMB36_X3Y24:RAMB36_X4Y25, RAMB36_X3Y22:RAMB36_X5Y23,RAMB36_X3Y20:RAMB36_X4Y21, RAMB36_X3Y18:RAMB36_X5Y19, RAMB36_X3Y6:RAMB36_X4Y17;
INST "*/inbuf_inst/*" AREA_GROUP="inbuf";
\end{lstlisting}
ISE 10.1 is not able to generate a working hardware implementation that
meets timing requirements. This is caused by the high block ram utilization
and the high frequency of the LVDS transmitter (\Cref{sec:outbuf}), which is
near the possible maximum frequency of the FPGA. In order to achieve timing
goals floor planning has to be used. Floor planning also helps speeding up
the implementation process.\\
The easiest way to create a floor plan is to use \emph{planAhead}, draw
\emph{Pblocks} and assign primitives to those. One problem with this
approach is, that after this operation every instantiated primitive
is fixed and the floor plan has to be adapted after every change to the
design. This can be circumvented with adjusting the generated ucf file
to use wildcards instead of instance names. An example can be seen in
\Cref{lst:floorplan}.\\
If the processor is used the first column of slices must not be used, because
this area is reserved for the DDR memory interface. The constraints file
generated by EDK prevents logic from being place there, because this interface
is already pre routed. Since this approach has been used, because ISE 10.1 is
not able to route the design. This short coming has been fixed in a later
version.\\
An overview of the floor plan, describing the different areas, can be seen in
\Cref{fig:floorplan}.
\begin{figure}[h]
    \centering
    \begin{pspicture}(-2.5,0)(9.5,14.5)
        \rput[bl](0,0){\includegraphics*[bb=3 2 252 381]{floorplan.ps}}
        \rput(9.5,7.5){\rnode{GTXdesc}{\psframebox{GTX}}}
        \pnode(8.62,5.7){GTX1}
        \pnode(8.62,9.1){GTX2}
        \rput(2,14){\rnode{MEMdesc}{\psframebox{memories}}}
        \pnode(1.07,13.1){MEM1}
        \pnode(2.22,13.1){MEM2}
        \pnode(3.37,13.1){MEM3}
        \rput(5.95,14){\rnode{DSPdesc}{\psframebox{DSP blocks}}}
        \pnode(5.72,13.1){DSP1}
        \pnode(6.19,13.1){DSP2}
        \rput(-1.5,11){\rnode{PPCdesc}{\psframebox{PPC}}}
        \pnode(2.4,7.8){PPC}
        \rput(-1.5,9){\rnode{LOCdesc}{\psframebox{\shortstack{outdata LOC\\constraints}}}}
        \pnode(0.6,7.4){LOC}
        \rput(-1.5,5){\rnode{PADdesc}{\psframebox{\shortstack{outdata pads}}}}
        \pnode(0.2,6.8){PAD}
        \rput(-1.5,7.2){\rnode{AREAdesc}{\psframebox{\shortstack{outdata AREA\\constraints}}}}
        \pnode(0.7,7.2){AREA}
        \psset{arrows=->}
        \ncline{LOCdesc}{LOC}
        \ncline{PADdesc}{PAD}
        \ncline{AREAdesc}{AREA}
        \ncline{PPCdesc}{PPC}
        \ncline{MEMdesc}{MEM1}
        \ncline{MEMdesc}{MEM2}
        \ncline{MEMdesc}{MEM3}
        \ncline{DSPdesc}{DSP1}
        \ncline{DSPdesc}{DSP2}
        \ncline{GTXdesc}{GTX1}
        \ncline{GTXdesc}{GTX2}
    \end{pspicture}
    \caption{Floor plan. Major areas besides \emph{outdata}:
        \emph{outbuf\_mem\_*} and \emph{outbuf} keep outbuf memories close
        together and the related logic close to \emph{outbuf}, \emph{core} to contain
        the FFT IP, \emph{proc\_interface} for keeping the cpu interface
        close to the PPC block, and \emph{inbuf} for placing related logic and
        memories close to the GTX blocks.}
    \label{fig:floorplan}
\end{figure}
\section{Software}
To ease communication, achieve high speeds, and accomplish interoperability
with a high number of devices, ethernet has been chosen for communication.
Since this requires an IP stack and some accompanying tools an operating
system is run on the PPC. Linux has been chosen as kernel, because all
of the needed drivers are in mainline and the board is thoroughly tested
and well supported. Buildroot was used for building the tools and the
cross compilation tool chain.
\subsection{Buildroot}
\lstset{language=bash}
Buildroot can be downloaded from \url{http://buildroot.uclibc.org/}. The build
system is the same as the one used in the linux kernel and it is as easy to
create a full working embedded GNU/linux system as building a kernel.
\lstinline+make menuconfig+ launches the curses based configuration system. For this
board the minimal settings needed are:
\begin{description}
    \item[Target Architecture:] \hfill \\ PowerPC
    \item[Target Architecture Variant:] \hfill \\ 440
    \item[Use soft-float]
    \item[C library:] \hfill \\ glibc. 
\end{description}
uClibc can't be used because of a bug in uClibc/gcc which causes the error
``\emph{undefined reference to `copysignl'}'' during compilation.
This would create a bare minimal base system with a minimal busybox
environment and no kernel. The build can be invoked with \lstinline+make+, which
downloads all needed software, builds a cross compilation tool chain and 
creates the target image.\\
For this project all the needed configuration and additional software is
hosted on \url{https://github.com/notti/bak-buildroot}. To rebuild the image:
\begin{enumerate}
    \item Download buildroot from \url{http://buildroot.uclibc.org/}. This
        project has been tested with version 2014.08. Newer versions might
        need some tweaking.
    \item Extract it.
    \item Download \url{https://github.com/notti/bak-buildroot}.
    \item Extract this to a different directory.
\end{enumerate}
Finally execute the following from within the buildroot directory:
\begin{lstlisting}[language=bash,basicstyle=\ttfamily\tiny]
make BR2_EXTERNAL=<path to bak-buildroot> ml507_defconfig
make BR2_EXTERNAL=<path to bak-buildroot>
\end{lstlisting}
This generates the whole image and linux kernel. After invoking \lstinline+make+,
buildroot remembers the \lstinline+BR2_EXTERNAL+ setting. Customizations can be
done with \lstinline+make menuconfig+.\\
After a successful run, the target image and kernel is in \emph{output/images}.
In \emph{output/host} is the installation of the cross tool chain. This can be
used by adding \emph{output/host/usr/bin} to \lstinline+PATH+.
\subsection{Linux Kernel}
All the necessary drivers are already in the mainline kernel and therefor an unmodified
kernel tree from \url{https://www.kernel.org/} can be used. The latest version this
was tested with is 3.16.2, which is also included in buildroot 2014.08.
These following settings are the minimum needed to get the kernel up and running:
\begin{description}
    \item[Processor Type] \hfill \\ AMC 44x, 46x, 47x
    \item[Generic Xilinx Virtex 5 FXT board support]
    \item[Math emulation]
    \item[Initial kernel command string] \hfill \\
        console=ttyS0,115200\footnote{This needs to be adjusted to the used
        serial speed and the same as the buildroot image is using.}
        root=/dev/xsa2\footnote{/dev/xsa is the device node of the SystemAce CF
        Card.}
    \item[Xilinx SystemAce support]
    \item[Xilinx LL TEMAC (LocalLink Tri-mode Ethernet MAC) driver]
    \item[Drivers for Marvell PHYs]
    \item[Support for MMIO device-controlled MDIO bus multiplexers]
    \item[8250/16550 and compatible serial support]
    \item[Console on 8250/16550 and compatible serial port]
    \item[Serial port on Open Firmware platform bus]
\end{description}
As boot loader the in kernel simpleImage can be be used. This simple boot
loader uncompresses the kernel in memory and uses the UART interface to
present a kernel command line prompt. After a timeout or pressing the
return key the kernel is booted. This bootloader needs the devicetree
file from \Cref{sec:processor}. This file needs to be present in the
directory \emph{arch/powerpc/boot/dts/} below the kernel source
directory. The kernel image can be built with the following command
line:
\begin{lstlisting}[language=bash,basicstyle=\ttfamily\tiny]
make simpleImage.<dts name>
\end{lstlisting}
\subsection{Kernel Module}
\lstset{language=C,escapeinside={(*}{*)}}
\begin{lstlisting}[float,language=C,caption={Basic elements of a platform driver module},label=src:pdriver,basicstyle=\ttfamily\tiny]
...

struct emce_device { (*\label{deviceData}*)
	struct cdev cdev;
	dev_t dev;
	unsigned long reg_start;
	unsigned long reg_size;
	void __iomem *base_address;
	spinlock_t register_lock;
	struct user_mem mem[USER_MEM];
	unsigned int irq;
	struct kernfs_node *int_nodes[16];
};

...

static int emce_of_probe(struct platform_device *ofdev) { (*\label{probe}*)
	...
}

static int emce_of_remove(struct platform_device *of_dev) { (*\label{remove}*)
	...
}

static const struct of_device_id emce_of_match[] = { (*\label{match}*)
	{ .compatible = "xlnx,proc2fpga-3.00.b", },
	{ /* end of list */ },
};

MODULE_DEVICE_TABLE(of, emce_of_match); (*\label{dt}*)

static struct platform_driver emce_of_driver = { (*\label{platformDriver}*)
	.driver = {
		.name = DRIVER_NAME,
		.owner = THIS_MODULE,
		.of_match_table = emce_of_match,
	},
	.probe = emce_of_probe,
	.remove = emce_of_remove,
};

module_platform_driver(emce_of_driver); (*\label{pd}*)

MODULE_LICENSE("GPL"); (*\label{macros}*)
MODULE_AUTHOR("Gernot Vormayr <notti@fet.at");
MODULE_DESCRIPTION("driver for custom fpga interface");
\end{lstlisting}
Controlling the hardware from within the linux operating system can be
achieved with by either mmaping \emph{/dev/mem} or writing a kernel module.
mmap has the disadvantage that interrupts can not be used and concurrent
access leads to undefined behaviour. Another disadvantage is that directly
using \emph{/dev/mem} is dangerous and can easily lead to system crashes.
Furthermore a kernel module can provide an easy api that can be easily
used by any programming language, shell script or even echo. Because of
these advantages the following kernel module has been implemented.\\
To keep the driver as generic as possible a platform device driver was
written. Platform device drivers are matched against the device tree and
the kernel takes care of invoking the \lstinline+probe+ and
\lstinline+remove+ functions. Besides the common needed macros in
\Cref{src:pdriver} \Cref{macros} only a \lstinline+struct+ describing the
driver (\Cref{platformDriver}), a list of device names (\Cref{match}),
and the accompanying platform driver macros in \Cref{dt,pd} are needed.
These macros create the module instantiation code and take care of
everything. The only code that needs to be supplied are the bodies of
the probe and remove functions (\Cref{probe,remove}). The probe function
has to allocate memory for the device data stored in the
\lstinline+struct+ defined in \Cref{deviceData}, request access to the
memories, allocate devices and files, setup the interrupt function, and
initialize the interrupt registers. For freeing the resources, everything
has to be released again in the remove function. Omitting this prevents
reloading the driver.\\
The driver handles the three tasks:
\subsubsection{Memory access}
\lstset{language=C,escapeinside={<*}{*>}}
\begin{lstlisting}[float,language=C,caption={Character device},label=src:cdev,basicstyle=\ttfamily\tiny]
ssize_t mem_read (struct file *file, char __user *buf,
		size_t count, loff_t *ppos) { <*\label{mem_read}*>
	struct user_mem *mem = file->private_data;

	if(*ppos >= mem->size)  <*\label{mem_read_checka}*>
		return 0; //EOF

	if(*ppos + count >= mem->size)
		count = mem->size - *ppos; <*\label{mem_read_checkb}*>

	if(copy_to_user(buf, mem->base_address+*ppos, count)) <*\label{mem_read_copy}*>
		return -EFAULT;

	*ppos+=count; <*\label{mem_read_advance}*>
	return count;
}

ssize_t mem_write(struct file *file, const char __user *buf,
		size_t count, loff_t *ppos) {
	...
}

static int mem_open(struct inode *inode, struct file *file) { <*\label{mem_open}*>
	struct emce_device *edev;

	if(MINOR(inode->i_rdev)>=USER_MEM)
		return -ENODEV;

	edev = container_of(inode->i_cdev, struct emce_device, cdev);
	file->private_data = &edev->mem[MINOR(inode->i_rdev)];
	return 0;
}

static loff_t mem_lseek(struct file *file, loff_t offset, int orig) {
	...
}

...

static const struct file_operations emce_fops = { <*\label{fops}*>
	.owner = THIS_MODULE,
	.read = mem_read,
	.write = mem_write,
	.open = mem_open,
	.mmap = mem_mmap,
	.llseek = mem_lseek,
};

static struct class *emce_class;

...

        cdev_init(&edev->cdev, &emce_fops); <*\label{cdev_init}*>
	edev->cdev.owner = THIS_MODULE;
	kobject_set_name(&edev->cdev.kobj, "mem");
	if(cdev_add(&edev->cdev, edev->dev, USER_MEM)) {
		...
	}

	emce_class = class_create(THIS_MODULE, DRIVER_NAME);
	if(IS_ERR(emce_class))
		...

	for(minor = 0; minor < USER_MEM; minor++)
		device_create(emce_class, dev, MKDEV(MAJOR(edev->dev), minor),
				NULL, "emce%d", minor); <*\label{cdev_init_end}*>
\end{lstlisting}

The memories from \module{inbuf}, \module{outbuf} and \module{H} are exposed
to user space via character devices\cite[p. 42ff]{ldd}. During
\lstinline+probe+ the memory ranges are reserved with
\lstinline+request_mem_region+\cite[p. 249]{ldd} (only one driver is allowed
to handle a memory region at a time) and mapped to a virtual address with
\lstinline+ioremap+\cite[p. 250]{ldd} (only virtual memory can be accessed).
Character devices look to user space as if they were normal files which can
be opened, seeked, read and written. All those operations are possible on
the byte level, hence the name. The following paragraphs reference lines from
\Cref{src:cdev}.\\
To set up a character device a
\lstinline+struct_file_operations+\cite[p. 54]{ldd} is needed. This structure
has an entry for every possible operation, which in turn is a function pointer
which need to point to the implementation. Not implemented functions need to
point to \lstinline+NULL+, which is taken care of by leaving out the fields
in the GNU style initialization seen in \Cref{fops}. One of those functions
is \emph{close}, which is not needed by this driver, because only memory is
accessed and so there is virtually no setup done on \emph{open} and therefor
no cleanup needs to happen. Allocating the necessary
memory and initializing everything is taken care of by the functions in
\Crefrange{cdev_init}{cdev_init_end}. The most important functions are
\lstinline+cdev_init+\cite[p. 56]{ldd} which allocates the driver structure
and initializes the operations, and \lstinline+device_create+ which allocates
and creates the actual devices. Every device needs a major and a minor number,
which are used to identify which device a file belongs to. In this example the
major number is automatically allocated and assigned and the minor number
corresponds to the internal memory number. The rest of the code is boilerplate
which causes the kernel to create the device nodes \emph{/dev/emce0} to
\emph{/dev/emce3}. Without this routines these nodes would need to be created
manually with \emph{mknod}.\\
If user space calls the functions open on one of those device nodes, the
kernel deduces from the major number, that this driver is meant and calls
\lstinline+mem_open+ in \Cref{mem_open}. As mentioned earlier no setup needs
to be done here. This function only checks if the minor number, which tells
the driver which of the memories is meant, is in range and then assigns the
memory area to the private data of the inode\cite[p. 54]{ldd}. Storing this
information within the inode is not actually neede, but without this the
other functions would need to fetch this data on every invocation.\\
Every operation on an opened device is in turn handled by the rest of the
functions which are quiet similar and so only \lstinline+mem_read+ in
\Cref{mem_read} is explained. User space provides a buffer to write to,
and a number of bytes to read (\lstinline+count+). The current file
position is provided with \lstinline+ppos+. Since the requested information
is actually in memory, the only things that need to be done are:
\begin{enumerate}
    \item Boundary Check (\Crefrange{mem_read_checka}{mem_read_checkb}).
    \item Copy the data to user space (\Cref{mem_read_copy}). This must be
        done with the \lstinline+copy_*_user+ macros. Kernel memory and
        user memory must never be mixed!
    \item Advance the file position (\Cref{mem_read_advance}).
    \item Return the number of bytes actually read.
\end{enumerate}
Another additional function which is not absolutely necessary, but can speed
things up, is \emph{mmap} which won't be explained here, but can be looked up
in the sources.
\subsubsection{Register access}
Access to the registers is provided via files in sysfs\cite{sysfs}. To ease
user space handling and follow the spirit of sysfs with one setting per file,
every single flag from \Crefrange{reg0}{reg5} is implemented as a file in
sysfs. The linux kernel provides a very easy to use interface. A sysfs file
can be allocated with a \lstinline+device_attribute+ structure, which needs
a function pointer to a \emph{show} and \emph{store} function, access rights,
and a file name. Setting one of the function pointers to zero, disables read
and or write access. \emph{Show} is called if user space reads from a file 
and \emph{store} if user space writes to a file. Since these files are
supposed to represent a single value there are no \emph{open} and \emph{close}
functions or file positions. \emph{Show} is given a pointer to a single page
that has to be filled and the number of written bytes returned, where is
\emph{store} receives a pointer to a buffer with the user supplied data and
a size parameter.\\
The flags are assigned to attribute groups which in turn represent
subdirectories below the driver directory. For easier handling the driver
has two standard functions and some accompanying macros which can handle
flags in a bit field:
\begin{description}
    \lstitem{FPGA_FLAG(base, name, access, register, bit, length)}
        This macro creates a flag within the group \emph{base} with the name
        \emph{name} and access right \emph{access}. \emph{Register} represents
        the base address, \emph{bit} the bit number within the register and
        \emph{length} the bit width of the flag. \lstinline+_+ represents the
        base directory of the driver.
    \lstitem{FPGA_FLAGM(..., max)}
        This macro has the same arguments as \lstinline+FPGA_FLAG+ with an
        additional \emph{max}, which is the maximum value + 1, that is allowed.
    \lstitem{FPGA_FLAGC(..., show, store)}
        Same as \lstinline+FPGA_FLAGM+ but with the two additional
        arguments \lstinline+show+ and \lstinline+store+ so custom functions
        can be used.
\end{description}
Writing to the memory mapped registers is done with a read modify write cycle,
that is protected by a spin lock. This spin lock isn't actually needed by a
uni processor system and gets optimized away during the compilation. Since
one day this might run on another FPGA which could have a SMP system it is
advised to keep the spin lock.
\subsubsection{Interrupts}
These are reported to user space with sysfs files. The \emph{show} function
returns just an empty line. User space has to read the file and then wait for
an event with \emph{poll}. This function will block until an event is
triggered with \lstinline+sysfs_notify+. The interrupt service routine is
therefor very simple and does the following things:
\begin{enumerate}
    \item Read the interrupt register.
    \item Write to the interrupt register to clear the interrupts.
    \item Call \lstinline+sysfs_notify_dirent+ on the appropriate files. Since
        \lstinline+sysfs_notify+ would need to do a path lookup which is not
        allowed from within interrupts (the path lookup needs a mutex), the
        \lstinline+_dirent+ version is used instead.
    \item Return \lstinline+IRQ_HANDLED+ to signal the kernel, that the
        interrupt handling was successful.
\end{enumerate}
\Cref{sec:sysfs} presents examples in detail on how to access these events.
\section{Boostrap}
\subsection{build image}
\subsection{debug/test hw}
\subsection{build complete image}
\section{usage}
what how and why
\subsection{boot}
\subsection{sysfs}
\label{sec:sysfs}
\subsection{webinterface}


\clearpage
\appendices
\section{Registers}
\begin{register}{h}{reg(0)}{0x00}%
    \label{reg0}%
    \regfield{rec\_rst}{1}{31}{0}%
    \regfield{Reserved}{4}{27}{0}%
    \regfield{rec\_stream\_valid}{1}{26}{0}%
    \regfield{Reserved}{1}{25}{0}%
    \regfield{rec\_input\_select}{1}{24}{0}%
    \regfield{Reserved}{10}{14}{0}%
    \regfield{rec\_data\_valid(1)}{1}{13}{0}%
    \regfield{rec\_rxeqmix(1)}{2}{11}{0}%
    \regfield{rec\_descramble(1)}{1}{10}{1}%
    \regfield{rec\_polarity(1)}{1}{9}{1}%
    \regfield{rec\_enable(1)}{1}{8}{1}%
    \regfield{Reserved}{2}{6}{0}%
    \regfield{rec\_data\_valid(0)}{1}{5}{0}%
    \regfield{rec\_rxeqmix(0)}{2}{3}{0}%
    \regfield{rec\_descramble(0)}{1}{2}{1}%
    \regfield{rec\_polarity(0)}{1}{1}{1}%
    \regfield{rec\_enable(0)}{1}{0}{1}%
    \reglabel{Reset}\regnewline%
    \begin{regdesc}\begin{reglist}[rec\_descramble(n)]
        \item[rec\_rst] Writing a 1 to this bit resets the whole receiver.
        \item[rec\_stream\_valid] 1 if stream is valid.
        \item[rec\_input\_select] With this the data stream of receivers 0 or 1
            can be selected.
        \item[rec\_data\_valid(n)] 1 if the data received by n is valid.
        \item[rec\_rxeqmix(n)] Controls the equalizer of the receiver
            \cite[p. 165f]{gtx}.
        \item[rec\_descramble(n)] Turns on the descrambler (see
            \Cref{sec:inbuf}).
        \item[rec\_polarity(n)] Sets the polarity of the LVDs pair.
        \item[rec\_enable(n)] Enable/Disable transceiver. Disabling unneeded
            transceivers saves power.
    \end{reglist}\end{regdesc}
\end{register}
\begin{register}{h}{reg(1)}{0x04}%
    \label{reg1}%
    \regfield{avg\_rst}{1}{31}{0}%
    \regfield{Reserved}{3}{28}{0}%
    \regfield{avg\_err}{1}{27}{0}%
    \regfield{avg\_active}{1}{26}{0}%
    \regfield{avg\_width}{2}{24}{0}%
    \regfield{trig\_rst}{1}{23}{0}%
    \regfield{auto\_rst}{1}{22}{0}%
    \regfield{auto\_single}{1}{21}{0}%
    \regfield{auto\_run}{1}{20}{0}%
    \regfield{trig\_int}{1}{19}{0}%
    \regfield{trig\_arm}{1}{18}{0}%
    \regfield{Reserved}{1}{17}{0}%
    \regfield{trig\_type}{1}{16}{0}%
    \regfield{depth}{16}{0}{0}%
    \reglabel{Reset}\regnewline%
    \begin{regdesc}\begin{reglist}[auto\_single]
        \item[avg\_rst] Writing a 1 to this bit resets \module{average\_mem}.
        \item[avg\_err] 1 if the data stream became invalid during data
            acquisition. Gets on next trigger event.
        \item[avg\_active] 1 during data acquisition.
        \item[avg\_width] Number of samples over which averaging is done:
            \begin{itemize}
                \item[0:] off.
                \item[1:] 2 samples.
                \item[2:] 4 samples.
                \item[3:] 8 samples.
            \end{itemize}
        \item[auto\_rst] Writing a 1 to this bit resets \module{auto}.
        \item[auto\_single] Does a single automatic mode cycle
            (acquire, convolute, switch output).
        \item[auto\_run] Automatic mode.
        \item[trig\_rst] Writing a 1 to this bit resets \module{trigger}. 
        \item[trig\_int] Writing a 1 to this bit manually triggers the
            internal trigger.
        \item[trig\_arm] Writing a 1 to this bit arms the trigger.
        \item[trig\_type] 0: Internal trigger. 1: External trigger.
        \item[depth] Number of samples to acquire. Can be range 1 - 49152.
    \end{reglist}\end{regdesc}
\end{register}
\begin{register}{h}{reg(2)}{0x08}%
    \label{reg2}%
    \regfield{Reserved}{4}{28}{0}%
    \regfield{core\_scale\_schi}{12}{16}{011010101010}%
    \regfield{Reserved}{4}{12}{0}%
    \regfield{core\_scale\_sch}{12}{0}{011010101010}%
    \reglabel{Reset}\regnewline%
    \begin{regdesc}\begin{reglist}[core\_scale\_schi]
        \item[core\_scale\_schi] Scaling schedule for iFFT run (See
            \cite[p. 24]{xilinx_fft}).
        \item[core\_scale\_sch] Scaling schedule for FFT run (See
            \cite[p. 24]{xilinx_fft}).
    \end{reglist}\end{regdesc}
\end{register}
\begin{register}{h}{reg(3)}{0x0C}%
    \label{reg3}%
    \regfield{core\_rst}{1}{31}{0}%
    \regfield{Reserved}{1}{30}{0}%
    \regfield{core\_circular}{1}{29}{0}%
    \regfield{core\_ov\_cmul}{1}{28}{0}%
    \regfield{core\_ov\_ifft}{1}{27}{0}%
    \regfield{core\_ov\_fft}{1}{26}{0}%
    \regfield{core\_start}{1}{25}{0}%
    \regfield{core\_iq}{1}{24}{0}%
    \regfield{Reserved}{3}{21}{0}%
    \regfield{core\_n}{5}{16}{00011}%
    \regfield{core\_scale\_cmul}{2}{14}{0}%
    \regfield{Reserved}{2}{12}{0}%
    \regfield{core\_L}{12}{0}{0}%
    \reglabel{Reset}\regnewline%
    \begin{regdesc}\begin{reglist}[core\_scale\_cmul]
        \item[core\_rst] Writing a 1 to this bit resets \module{core}.
        \item[core\_circular] Set to 1 for circular convolution.
        \item[core\_ov\_cmul] 1 if complex multiplication has overflown.
        \item[core\_ov\_ifft] 1 if iFFT has overflown.
        \item[core\_ov\_fft] 1 if FFT has overflown.
        \item[core\_start] Write 1 to this bit to start a convolution run.
        \item[core\_iq] Set to 1 for I/Q demodulation.
        \item[core\_n] Transform size in $\log_2(nfft)$. Valid values:
            \begin{itemize}
                \item[3:] 8
                \item[4:] 16
                \item[5:] 32
                \item[6:] 64
                \item[7:] 128
                \item[8:] 256
                \item[9:] 512
                \item[10:] 1024
                \item[11:] 2048
                \item[12:] 4096
            \end{itemize}
        \item[core\_scale\_cmul] Scaling schedule for complex multiplication:
            \begin{itemize}
                \item[0:] $>> 17$.
                \item[1:] $>> 16$.
                \item[2:] $>> 15$.
                \item[3:] $>> 14$.
            \end{itemize}
        \item[core\_L] $L$. See \Cref{sec:core}.
    \end{reglist}\end{regdesc}
\end{register}
\begin{register}{h}{reg(4)}{0x10}%
    \label{reg4}%
    \regfield{tx\_mulq}{16}{16}{0}%
    \regfield{tx\_muli}{16}{0}{0}%
    \reglabel{Reset}\regnewline%
    \begin{regdesc}\begin{reglist}[tx\_mulq]
        \item[tx\_mulq] See \Cref{sec:outbuf}.
        \item[tx\_muli] See \Cref{sec:outbuf}.
    \end{reglist}\end{regdesc}
\end{register}
\begin{register}{h}{reg(5)}{0x14}%
    \label{reg5}%
    \regfield{tx\_shift}{4}{28}{0}%
    \regfield{Reserved}{2}{26}{0}%
    \regfield{tx\_ovfl}{1}{25}{0}%
    \regfield{tx\_sat}{1}{24}{1}%
    \regfield{tx\_rst}{1}{23}{0}%
    \regfield{Reserved}{3}{20}{0}%
    \regfield{tx\_resync}{1}{19}{0}%
    \regfield{tx\_toggle}{1}{18}{0}%
    \regfield{tx\_dc\_balance}{1}{17}{0}%
    \regfield{tx\_deskew}{1}{16}{0}%
    \regfield{tx\_frame\_offset}{16}{0}{0}%
    \reglabel{Reset}\regnewline%
    \begin{regdesc}\begin{reglist}[tx\_frame\_offset]
        \item[tx\_shift] Scaling schedule for complex multiplication:
            \begin{itemize}
                \item[0:] $>> 17$.
                \item[1:] $>> 16$.
                \item[2:] $>> 15$.
                \item[3:] $>> 14$.
                \item[4:] $>> 13$.
                \item[5:] $>> 12$.
                \item[6:] $>> 11$.
                \item[7:] $>> 10$.
                \item[8:] $>> 9$.
                \item[9:] $>> 8$.
                \item[10:] $>> 7$.
                \item[11:] $>> 6$.
                \item[12:] $>> 5$.
                \item[13:] $>> 4$.
                \item[14:] $>> 3$.
                \item[15:] $>> 2$.
            \end{itemize}
        \item[tx\_ovfl] 1 if an overflow occurred in the complex multiplier.
            Write a 0 to this bit to reset the overflow.
        \item[tx\_sat] Enable saturation for complex multiplier.
        \item[tx\_rst] Writing a 1 to this bit resets \module{outbuf}.
        \item[tx\_resync] Resynchronizes the sample output to
            tx\_frame\_offset on the internal frame clock.
        \item[tx\_toggle] Writing a 1 to this bit switches the output buffers.
        \item[tx\_dc\_balance] Turns DC balance on for the output. (See
            \cite[p. 11]{ds90cr485}).
        \item[tx\_deskew] Writing a 1 to this bit starts a deskew cycle. (See
            \cite[p. 12f]{ds90cr485}).
        \item[tx\_frame\_offset] Frame offset.
    \end{reglist}\end{regdesc}
\end{register}
\begin{register}{h}{intr}{0x220}%
    \label{intr}%
    \regfield{Reserved}{19}{13}{0}%
    \regfield{auto\_stop}{1}{12}{0}%
    \regfield{auto\_start}{1}{11}{0}%
    \regfield{tx\_ovfl}{1}{10}{0}%
    \regfield{tx\_toggled}{1}{9}{0}%
    \regfield{core\_done}{1}{8}{0}%
    \regfield{avg\_done}{1}{7}{0}%
    \regfield{trigd}{1}{6}{0}%
    \regfield{stream\_invalid}{1}{5}{0}%
    \regfield{stream\_valid}{1}{4}{0}%
    \regfield{rec1\_invalid}{1}{3}{0}%
    \regfield{rec1\_valid}{1}{2}{0}%
    \regfield{rec0\_invalid}{1}{1}{0}%
    \regfield{rec0\_valid}{1}{0}{0}%
    \reglabel{Reset}\regnewline%
\end{register}

\clearpage
\section{Sources}
\begin{description}
    \item[ISE project, VHDL sources:] \hfill \\
        \url{https://github.com/notti/bak-hardware}
    \item[External buildroot settings, patches, linux kernel config:] \hfill \\
        \url{https://github.com/notti/bak-buildroot}
    \item[Linux kernel module, webinterface:] \hfill \\
        \url{https://github.com/notti/bak-software}
    \item[buildroot:] \hfill \\
        \url{http://buildroot.uclibc.org/}
    \item[linux:] \hfill \\
        \url{https://www.kernel.org/}
    \item[This documentation:] \hfill \\
        \url{https://github.com/notti/bak-doc}
\end{description}

\bibliographystyle{IEEEtran}
\begingroup
\raggedright
\sloppy
\bibliography{IEEEabrv,literature}
\endgroup

\end{document}
